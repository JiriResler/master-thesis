\chapter{Design}
Now we will design how to implement what we have defined in the analysis chapter.
Our application is a single-page web application.

\section{Architecture}
\begin{figure}[h]
  \centering
  \includegraphics[width=0.62\linewidth]{master-thesis/img/architecture_data_flow.pdf}
  \caption{Application's data flow diagram}
\end{figure}

\section{Technological stack}
% popisat dostupne technologie - napisat ktore som si vybral a preco
There are various options out there for what technologies we can use to implement our application.
This section contains an overview of available tools and which ones we chose and why.

\subsection{Front-end}
Nowadays, there are many frameworks for building web applications.
The most popular are listed below with their pros and cons.

\subsubsection*{Angular}
Angular is a TypeScript-based, free and open-source web application framework.
It uses concepts of single-page application in which UI is delivered in the beginning of application request and later only data is requested which makes single-page application applications fast.
Projects in Angular are structured into Modules, Components, and Services. 
Each Angular application has at least one root component and one root module.
Each component in Angular contains a Template, a class that defines the application logic, and MetaData. 
The metadata for a component tells Angular where to find the building blocks that it needs to create and present its view.
Angular templates are written in HTML but can also include Angular template syntax with special directives to output reactive data and render multiple elements.
Services in Angular are used by Components to delegate business-logic tasks such as fetching data or validating input.

\subsubsection*{React.js}
React is a free and open-source front-end JavaScript library for rendering user interfaces.
Applications written in React are built from individual pieces called components.
These components are modular and reusable.
% A component is a piece of the UI that has its own logic and appearance.
% React components receive data and return what should appear on the screen. 
% They can be passed new data in response to an interaction, like when the user types into an input. 
% React will then update the screen to match the new data.
React adheres to the declarative programming paradigm. 
Developers design views for each state of an application, and React updates and renders components when data changes.
A notable feature is the use of a virtual Document Object Model, or Virtual DOM. 
React creates an in-memory data-structure cache, computes the resulting differences on a re-render, and then updates the browser's displayed DOM efficiently. 
% This process is called reconciliation. 
This allows the programmer to write code as if the entire page is rendered on each change, while React only renders the components that actually change. 
This selective rendering provides a major performance boost.

\subsubsection*{Vue.js}

As for programming languages, we consider two options. 

\begin{description}
  \item[JavaScript] 
  \item[TypeScript] 
\end{description}

% react router

\subsection{nasadenie}
%  Vite, npm
% create react app, webpack, yarn

\subsection{Responsive design}
\begin{description}
  \item[Bootstrap] 
  \item[Material UI] 
\end{description}

\subsection{Persistence}
% Solid pods

\subsection{Testing}
% https://legacy.reactjs.org/docs/testing-environments.html

\subsection{Documentation}
% (GitHub markdown)

\section{Wireframes}
\begin{figure}[h]
  \centering
  \includegraphics[width=0.62\linewidth]{master-thesis/img/wireframes/guest_profile_editor/login.pdf}
  \caption{Guest profile editor login screen}
\end{figure}

\begin{figure}[h]
  \centering
  \includegraphics[width=0.62\linewidth]{master-thesis/img/wireframes/guest_profile_editor/my_allergies.pdf}
  \caption{Guest profile editor allergies screen}
\end{figure}

\begin{figure}[h]
  \centering
  \includegraphics[width=0.62\linewidth]{master-thesis/img/wireframes/guest_profile_editor/my_diets.pdf}
  \caption{Guest profile editor diets screen}
\end{figure}

\begin{figure}[h]
  \centering
  \includegraphics[width=0.62\linewidth]{master-thesis/img/wireframes/guest_profile_editor/my_food_preferences.pdf}
  \caption{Guest profile editor food preferences screen}
\end{figure}

\begin{figure}[h]
  \centering
  \includegraphics[width=0.62\linewidth]{master-thesis/img/wireframes/guest_profile_editor/my_favorite_restaurants.pdf}
  \caption{Guest profile editor favorite restaurants screen}
\end{figure}

\begin{figure}[h]
  \centering
  \includegraphics[width=0.62\linewidth]{master-thesis/img/wireframes/guest_profile_editor/menu_expanded.pdf}
  \caption{Guest profile editor screen with expanded menu}
\end{figure}

\listoftodos