\chapter{Existing solutions}
In this chapter we will discuss currently available solutions in our problem domain.
First, each application is briefly introduced.
Second, we state use cases by which we later compare the applications.
Last, we describe how each application fulfills these use cases.

\section{Overview of existing similar applications}
Now we will briefly introduce currently available applications.

% osnova popisovania aplikacie:
% uvodne predstavenie, ako sa s nou pracuje, vyhody (co aplikacia nesplna popisat v comparison of apps), dalsie zaujimavosti

\subsection*{Allergy Menu}
  Allergy menu (link in footnote) allows a restaurant employee to maintain a mobile accessible and up-to-date menu that customers can tailor to their food preferences.
  
  A restaurant employee first creates a new menu and adds categories to it, e.g. "Starters" or "Desserts".
  Dishes are then added to each category of the menu.
  A dish contains allergen information as well as flags whether it is suitable for vegans and vegetarians.
  The restaurant employee can also add calories to the dish.

  When a restaurant employee wants to change something in a dish, they can create a copy of the dish which is hidden on the published menu.
  After they finish changes, the dish can be marked as "live", making it appear on the menu with updated content. 

  A dish can optionally contain internal notes, with the ability to upload photographs of products used within the dish.
  The application sends and e-mail regularly to review a menu with all the allergy information.
  An existing menu can be imported to the application in the CSV format.
  Allergy Menu provides an API for food suppliers, allowing them to sync menu information directly into the application without manual intervention.
  
  A guest can interact with a published menu by choosing what allergens they want to avoid.
  The application then filters out items of the menu to meet the guest's preferences.
  A guest can also select an option that they are either a vegan or vegetarian.
  
  A unique restaurant code is used to identify restaurants. 
  This is what guests enter into the application when they want to browse a menu.
  A guest can also find a restaurant which uses Allergy Menu on a map.
  \todo[inline]{a restaurant can only have one menu published - mention in comparison}

  \begin{figure}[h]
    \centering
    \includegraphics[width=0.62\linewidth]{master-thesis/img/existing-applications-screenshots/allergy_menu_screenshot}
    \caption{The Allergy Menu application}
  \end{figure}
% end of \subsection

\subsection*{Allergen Checker}
  Allergen Checker (link in footnote) is an allergen management software which enables a restaurant employee to add allergen information to a menu.
  
  A menu is created in three steps.
  In the first step, the restaurant employee creates ingredients which are then stored in a database called the restaurant's "virtual food cupboard".
  During the second step, the restaurant employee creates dishes by specifying their ingredients.
  In the third step, the restaurant employee creates a menu by specifying its dishes.
  
  The application provides a pre-defined list of basic ingredients in its database.
  An ingredient has information about what allergen it contains and also what allergens it may contain. 
  The restaurant employee can also copy information from the ingredient's packaging which will be then displayed in a dish's description in the menu.
  
  Allergen Checker allows for categorizing of dishes and menus. 
  Categories are thought of as a file system for dishes and menus.
  This is convenient when the restaurant employee needs to search for a specific menu or dish.

  \begin{figure}[h]
    \centering
    \includegraphics[width=0.62\linewidth]{master-thesis/img/existing-applications-screenshots/allergen_checker_menu_screenshot}
    \caption{The Allergen Checker application}
  \end{figure}
% end of \subsection

\subsection*{BigZpoon}
  Cloud based solution

  B2B SaaS venture

  ordering system

  Menu website can be integrated into the restaurants website

  Menu has 3 parts - ok to eat, ok to eat with modifications, not ok to eat

  The menu contains information why a menu is not ok to eat

  Detailed nutrition facts about meals

  As guest changes ingredients of what they want to eat the nutrition facts table changes too

  Guest can set nutritional goals

  App tracks nutrition of whole meal

  Can order online

  Creating a menu item comprises of specifying all modifications

  Usage analytics and insights 

  reviews (stars) of restaurants

  “Once the guests have entered their preferences, our AI driven software guides them to choose the right meal options by clearly marking the items and choices that fit their restrictions.  Further, our real-time nutrition calculator keeps accurate track of all nutritional details for chosen meal items and alerts the guest when the goals are exceeded.”

  \begin{figure}[h]
    \centering
    \includegraphics[width=0.62\linewidth]{master-thesis/img/existing-applications-screenshots/bigzpoon_screenshot}
    \caption{The BigZpoon application}
  \end{figure}
% end of \subsection

\subsection*{Tenkites}
  can link to existing EAS and POS system
  information to tenkites and out to every point of sale
  can be connected to social media
  optimized for search engines
  dashboard insights

  \begin{figure}[h]
    \centering
    \includegraphics[width=0.62\linewidth]{master-thesis/img/existing-applications-screenshots/tenkites_menu_screenshot}
    \caption{The Tenkites application}
  \end{figure}
% end of \subsection

\subsection*{Menu Guide}
  no download of app needed

  you may choose to highlight food items that are dairy-free, vegetarian and vegan

  Additional menu choices (like halal, low-calorie, low GL) can be shown in the editable notes field.

  No internet of WiFi in your venue? No problem, simply download our offline menus app to a shared tablet (available for Android devices on Google Play) and your customers can browse your allergen menus no matter where they are!

  select allergens and the app highlights affected dishes

  \begin{figure}[h]
    \centering
    \includegraphics[width=0.62\linewidth]{master-thesis/img/existing-applications-screenshots/menu_guide_screenshot}
    \caption{The Menu Guide application}
  \end{figure}
% end of \subsection

\todo[inline]{add Hubl app and restaurantallergens.com}
% \subsection*{Hubl}  

  % \begin{figure}[h]
  %   \centering
  %   \includegraphics[width=0.62\linewidth]{master-thesis/img/existing-applications-screenshots/.png}
  %   \caption{The  application}
  % \end{figure}
% end of \subsection