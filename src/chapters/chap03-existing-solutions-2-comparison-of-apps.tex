We will now compare the applications.
Table x contains guest application use cases.
Table x contains restaurant application use cases.
Use cases were made up.

% Guest application use cases
G1 Display an interactive menu to the guest.
G2 Create a dietary profile of the guest.
G3 Hide items of a menu which the guest cannot eat.
G4 Highlight items of a menu which the guest cannot eat.
G5 Search for a restaurant.
G6 Translate a menu (to language of UI).
G7 Filters menu items based on whether they are vegan or vegetarian.
G8 Update the list of the guest's favorite restaurants.
G9 Let the user have control over their data.
G10 Let the guest specify where should their data be stored.
G11 Display currently served meals by the guest's favorite restaurants.

\begin{center}
  \begin{tabular}{| c | l |}
    \hline
    G1 & Allow a guest to create a dietary profile. \\
    \hline
    G2 & Display a personalized menu to a guest. \\
    \hline
    G3 & Allow a guest to filter out items of a menu which they cannot eat. \\
    \hline
    G4 & Display a menu after a quest scans a QR code on a printed menu. \\
    \hline
    G5 & Allow a guest to mark a restaurant as their favorite. \\
    \hline
    G6 & Allow a guest to control their personal data. \\
    \hline
    G7 & Translate a menu to the language of the UI. \\
    \hline
    G8 & Find restaurants near the guest based on their location. \\
    \hline
    G9 & Allow the guest to pay for a menu item. \\ % aplikácia umožňuje zaplatiť za účet v reštaurácii v rámci aplikácie
    \hline
    G10 & možnosť pracovať s aplikáciou neprihlásený \\
    \hline
  \end{tabular}
  \newline
\end{center}

% Restaurant application use cases
R1 Add allergen information to a menu.
R2 Set currency and weight units for a menu.
R3 Save menu items for future use.
R4 Set when a menu will be valid (active). 
R5 Provide pre-defined ingredients and dishes.
R6 Have multiple menus of a restaurant active at the same time.
R7 Generate a QR code for a menu.
R8 Provides templates for a new menu. 
R9 Switch between allergen names and numbers.
R10 Let the restaurant specify where should its data be stored.

\begin{center}
  \begin{tabular}{| c | l |}
    \hline
    % R1 & Automatically add allergens to an item of a menu. \\
    % \hline
    % R2 & Allow a restaurant employee to choose whether to use allergen names or numbers in a menu. \\
    % \hline
    % R3 & Provide templates for creating a new menu. \\
    % \hline
    % R4 & Allow a restaurant employee to reuse a previously created menu item. \\
    % \hline
    % R5 & Allow a restaurant employee to control their restaurant's data. \\    
    % \hline
    % R6 & weights of meals \\
    % \hline 
    % R7 & možnosť automaticky vygenerovať menu \\
    \hline
  \end{tabular}
  \newline
\end{center}

Use case G1 is supported by ...






Table x and x contain results of the comparison.

\todo[inline]{vertically center contents of the table - may be usable for all tables in the thesis} % https://tex.stackexchange.com/questions/7208/how-to-vertically-center-the-text-of-the-cells/611601#611601
\todo[inline]{make all columns have equal width}
\begin{center}
  \begin{tabular}{| l | c | c | c | c | c | c | c | c | c | c|}
    \hline 
      & G1 & G2 & G3 & G4 & G5 & G6 & G7 & G8 & G9 & G10 \\
    \hline
    Choose Well & \ding{52} & \ding{52} & \ding{52} & \ding{52} & \ding{52} & \ding{52} & \ding{52} & \ding{52} & \ding{52} & \ding{52} \\
    \hline
    % App2 & \ding{52} & \ding{56} & - \\
    % \hline
  \end{tabular}
  \newline
\end{center}

\begin{center}
  \begin{tabular}{| l | c | c | c | c | c | c | c |}
    \hline 
      & R1 & R2 & R3 & R4 & R5 & R6 & R7 \\
    \hline
    Choose Well & \ding{52} & \ding{52} & \ding{56} & \ding{56} & \ding{52} & \ding{56} & \ding{52} \\
    \hline
    % App2 & \ding{52} & \ding{56} & - \\
    % \hline
  \end{tabular}
  \newline
\end{center}

\ding{52} means that an application fully supports the use case \newline
\ding{56} means that an application does not support the use case \newline
\ding{115} means that an application supports the use case only partially \newline
- means that the use case is skipped as it is out of the application's scope \newline
