\section{Comparison of the applications}
We will now compare the applications.
Table x contains guest application use cases.
Table x contains restaurant application use cases.
Use cases were made up.

\todo[inline]{The application provides an overview of currently served meals}

% Guest application use cases
\begin{center}
  \begin{tabular}{| c | l |}
    \hline
    G1 & Display a menu with allergen information. \\
    \hline
    G2 & Highlight items of a menu which the guest cannot eat.  \\
    \hline
    G3 & Hide items of a menu which the guest cannot eat.  \\
    \hline
    G4 & Save the guest's dietary preferences.  \\
    \hline
    G5 & Translate a menu.  \\
    \hline
    G6 & Distinguish whether a menu item is vegan or vegetarian.  \\
    \hline
    G7 & Update the list of the guest's favorite restaurants.  \\
    \hline
    G8 & Let the guest specify where should their data be stored. \\ 
    \hline
  \end{tabular}
  \newline
\end{center}

Use case G1 is supported by all the applications.
Use case G2 is supported by Menu Guide, BigZpoon and Choose Well.
Use case G3 is supported by Allergy Menu, Menu Guide and Choose Well.
Use case G4 is supported by Menu Guide and Choose Well.
Use case G5 is supported by Menutech and Choose Well.
Use case G6 is supported by all applications except BigZpoon. 
% G7 - not all applications save information about guests
Use cases G7 and G8 are supported only by Choose Well.

% Restaurant application use cases
\begin{center}
  \begin{tabular}{| c | l |}
    \hline
    R1 & Add allergen information to a menu. \\
    \hline
    R2 & Save a menu item for future use. \\
    \hline
    R3 & Configure when a menu will be valid. \\
    \hline
    R4 & Provide pre-defined food ingredients. \\
    \hline
    R5 & Have multiple menus of a restaurant active at the same time. \\    
    \hline
    R6 & Generate a QR code for a menu. \\
    \hline 
    R7 & Provide design templates for a new menu. \\
    \hline 
    R8 & Switch between allergen names/icons and numbers. \\
    \hline 
    R9 & Let the restaurant employee specify where to store data. \\
    \hline
  \end{tabular}
  \newline
\end{center}

Use case R1 is supported by all the applications.
Use case R2 is supported by Menutech, Menu Guide and Choose Well.
% R3 Allergen checker - time as text in menu
Use case R3 is supported by Menutech, BigZpoon, Allergen Checker and Choose Well.
Use case R4 is supported by BigZpoon, Allergen Checker and Choose Well.
Use case R5 is supported by all except Allergy Menu.
Use case R6 is supported by Menutech, Menu Guide and Choose Well.
Use case R7 is supported by Menutech and Menu Guide.
Use cases R8 and R9 are supported only by Choose Well.

Table x and x contain results of the comparison.

\todo[inline]{vertically center contents of the table - may be usable for all tables in the thesis} % https://tex.stackexchange.com/questions/7208/how-to-vertically-center-the-text-of-the-cells/611601#611601
\todo[inline]{make all columns have equal width}
% Guest use cases
\begin{center}
  \begin{tabular}{| l | c | c | c | c | c | c | c | c | c | c |}
    \hline 
      & G1 & G2 & G3 & G4 & G5 & G6 & G7 & G8 \\
    \hline
    Menutech         & \ding{52} & \ding{56} & \ding{56} & \ding{56} & \ding{52} & \ding{52} & \ding{56} & \ding{56} \\
    \hline
    Allergy Menu     & \ding{52} & \ding{56} & \ding{52} & \ding{56} & \ding{56} & \ding{52} & \ding{56} & \ding{56}  \\
    \hline
    Menu Guide       & \ding{52} & \ding{52} & \ding{52} & \ding{52} & \ding{56} & \ding{52} & \ding{56} & \ding{56}  \\
    \hline
    BigZpoon Eagle   & \ding{52} & \ding{52} & \ding{56} & \ding{56} & \ding{56} & \ding{52} & \ding{56} & \ding{56}  \\
    \hline
    Allergen Checker & \ding{52} & \ding{56} & \ding{56} & \ding{56} & \ding{56} & \ding{56} & \ding{56} & \ding{56}  \\
    \hline
    Choose Well      & \ding{52} & \ding{52} & \ding{52} & \ding{52} & \ding{52} & \ding{52} & \ding{52} & \ding{52} \\
    \hline
  \end{tabular}
  \newline
\end{center}

% Restaurant use cases
\begin{center}
  \begin{tabular}{| l | c | c | c | c | c | c | c | c | c |}
    \hline 
      & R1 & R2 & R3 & R4 & R5 & R6 & R7 & R8 & R9 \\
    \hline
    Menutech         & \ding{52} & \ding{52} & \ding{52} & \ding{56} & \ding{52} & \ding{52} & \ding{52} & \ding{56} & \ding{56} \\
    \hline
    Allergy Menu     & \ding{52} & \ding{56} & \ding{56} & \ding{56} & \ding{56} & \ding{56} & \ding{56} & \ding{56} & \ding{56} \\
    \hline
    Menu Guide       & \ding{52} & \ding{52} & \ding{56} & \ding{56} & \ding{52} & \ding{52} & \ding{52} & \ding{56} & \ding{56} \\
    \hline
    BigZpoon Eagle   & \ding{52} & \ding{56} & \ding{52} & \ding{52} & \ding{52} & \ding{56} & \ding{56} & \ding{56} & \ding{56} \\
    \hline
    Allergen Checker & \ding{52} & \ding{56} & \ding{52} & \ding{52} & \ding{52} & \ding{56} & \ding{56} & \ding{56} & \ding{56} \\
    \hline
    Choose Well      & \ding{52} & \ding{52} & \ding{52} & \ding{52} & \ding{52} & \ding{52} & \ding{56} & \ding{52} & \ding{52} \\
    \hline
  \end{tabular}
  \newline
\end{center}

yes: \ding{52} means that an application supports the use case \newline
no: \ding{56} means that an application does not support the use case \newline

