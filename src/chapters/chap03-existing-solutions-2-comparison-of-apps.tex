We will now compare the applications.
Table x contains guest application use cases.
Table x contains restaurant application use cases.
Use cases were made up.

% Guest application use cases
\begin{center}
  \begin{tabular}{| c | l |}
    \hline
    G1 & Display a menu with allergen information. &  \\
    \hline
    G2 & Visually distinguish items of a menu which the guest cannot eat. &  \\
    \hline
    G3 & Create a dietary profile of the guest. &  \\
    \hline
    G4 & Search for a restaurant. \\
    \hline
    G5 & Translate a menu. &  \\
    \hline
    G6 & Filters menu items based on whether they are vegan or vegetarian. &  \\
    \hline
    G7 & Update the list of the guest's favorite restaurants. &  \\
    \hline
    G8 & Let the guest specify where should their data be stored. & \\ 
    \hline
  \end{tabular}
  \newline
\end{center}

% Restaurant application use cases
\begin{center}
  \begin{tabular}{| c | l |}
    \hline
    R1 & Add allergen information to a menu. \\
    \hline
    R2 & Save a menu item for future use. \\
    \hline
    R3 & Set when a menu will be valid. \\
    \hline
    R4 & Provide pre-defined ingredients and dishes. \\
    \hline
    R5 & Have multiple menus of a restaurant active at the same time. \\    
    \hline
    R6 & Generate a QR code for a menu. \\
    \hline 
    R7 & Provide templates for a new menu. \\
    \hline 
    R8 & Switch between allergen names/icons and numbers. \\
    \hline 
    R9 & Let a restaurant employee specify where should data be stored. \\
    \hline
  \end{tabular}
  \newline
\end{center}

Use case G1 is supported by all the applications.
Use case G2






Table x and x contain results of the comparison.

\todo[inline]{vertically center contents of the table - may be usable for all tables in the thesis} % https://tex.stackexchange.com/questions/7208/how-to-vertically-center-the-text-of-the-cells/611601#611601
\todo[inline]{make all columns have equal width}
% Guest use cases
\begin{center}
  \begin{tabular}{| l | c | c | c | c | c | c | c | c | c | c |}
    \hline 
      & G1 & G2 & G3 & G4 & G5 & G6 & G7 & G8 & G9 & G10 \\
    \hline
    Menutech & \ding{52} & - & - & - & - & - & - & - & - & - \\
    \hline
    Allergy Menu & \ding{52} & - & - & - & - & - & - & - & - & - \\
    \hline
    Menu Guide & \ding{52} & - & - & - & - & - & - & - & - & - \\
    \hline
    BigZpoon Eagle & \ding{52} & - & - & - & - & - & - & - & - & - \\
    \hline
    Allergen Checker & \ding{52} & - & - & - & - & - & - & - & - & - \\
    \hline
    Choose Well & \ding{52} & - & - & - & - & - & - & - & - & - \\
    \hline
  \end{tabular}
  \newline
\end{center}

% Restaurant use cases
\begin{center}
  \begin{tabular}{| l | c | c | c | c | c | c | c | c | c |}
    \hline 
      & R1 & R2 & R3 & R4 & R5 & R6 & R7 & R8 & R9 \\
    \hline
    Menutech & - & - & - & - & - & - & - & - & - \\
    \hline
    Allergy Menu & - & - & - & - & - & - & - & - & - \\
    \hline
    Menu Guide & - & - & - & - & - & - & - & - & - \\
    \hline
    BigZpoon Eagle & - & - & - & - & - & - & - & - & - \\
    \hline
    Allergen Checker & - & - & - & - & - & - & - & - & - \\
    \hline
    Choose Well & - & - & - & - & - & - & - & - & - \\
    \hline
  \end{tabular}
  \newline
\end{center}

\ding{52} means that an application fully supports the use case \newline
\ding{56} means that an application does not support the use case \newline
\ding{115} means that an application partially supports the use case \newline
- means that the use case is skipped as it is out of the application's scope \newline
