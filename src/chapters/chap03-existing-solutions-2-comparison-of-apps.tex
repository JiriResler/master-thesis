\section{Comparison of the applications}
First, we are going to specify use cases which an application such as ours should support.
\autoref{comparison-guest-use-cases} contains those use cases in which a guest uses the application.
\autoref{comparison-restaurant-use-cases} then contains those use cases in which a restaurant employee uses the application.

% Guest application use cases
\begin{table}[h]\centering
    \begin{tabular}{| c | l |}
      \hline
      G1 & Display a menu with allergen information. \\
      \hline
      G2 & Highlight items of a menu which the guest cannot eat.  \\
      \hline
      G3 & Hide items of a menu which the guest cannot eat.  \\
      \hline
      G4 & Save the guest's dietary preferences.  \\
      \hline
      G5 & Translate a menu.  \\
      \hline
      G6 & Distinguish whether a menu item is vegan or vegetarian.  \\
      \hline
      G7 & Update the list of the guest's favorite restaurants.  \\
      \hline
      G8 & Let the guest specify where should their data be stored. \\ 
      \hline
    \end{tabular}
    \caption{Application guest use cases}\label{comparison-guest-use-cases}
\end{table}

% Restaurant application use cases
\begin{table}[h]\centering
  \begin{tabular}{| c | l |}
    \hline
    R1 & Add allergen information to a menu. \\
    \hline
    R2 & Save a menu item for future use. \\
    \hline
    R3 & Configure when a menu will be valid. \\
    \hline
    R4 & Provide pre-defined food ingredients. \\
    \hline
    R5 & Have multiple menus of a restaurant active at the same time. \\    
    \hline
    R6 & Generate a QR code for a menu. \\
    \hline 
    R7 & Provide design templates for a new menu. \\
    \hline 
    R8 & Let the restaurant employee specify where to store data. \\
    \hline
  \end{tabular}
  \caption{Application restaurant use cases}\label{comparison-restaurant-use-cases}
\end{table}

\autoref{comparison-guest-use-cases-result} and \autoref{comparison-restaurant-use-cases-result} contain results of the comparison.
Value \ding{52} means that an application supports the given use case.
Value \ding{56} means that an application does not support the given use case. 

% Guest use cases
\begin{table}[h]\centering
  \begin{tabular}{| l | c | c | c | c | c | c | c | c |}
    \hline 
      & G1 & G2 & G3 & G4 & G5 & G6 & G7 & G8 \\
    \hline
    Menutech         & \ding{52} & \ding{56} & \ding{56} & \ding{56} & \ding{52} & \ding{52} & \ding{56} & \ding{56} \\
    \hline
    Allergy Menu     & \ding{52} & \ding{56} & \ding{52} & \ding{56} & \ding{56} & \ding{52} & \ding{56} & \ding{56}  \\
    \hline
    Menu Guide       & \ding{52} & \ding{52} & \ding{52} & \ding{52} & \ding{56} & \ding{52} & \ding{56} & \ding{56}  \\
    \hline
    BigZpoon Eagle   & \ding{52} & \ding{52} & \ding{56} & \ding{56} & \ding{56} & \ding{52} & \ding{56} & \ding{56}  \\
    \hline
    Allergen Checker & \ding{52} & \ding{56} & \ding{56} & \ding{56} & \ding{56} & \ding{56} & \ding{56} & \ding{56}  \\
    \hline
    Choose Well      & \ding{52} & \ding{52} & \ding{52} & \ding{52} & \ding{52} & \ding{52} & \ding{52} & \ding{52} \\
    \hline
  \end{tabular}
  \caption{Comparison of applications regarding use cases with guest}\label{comparison-guest-use-cases-result}
\end{table}

% Restaurant use cases
\begin{table}[h]\centering
  \begin{tabular}{| l | c | c | c | c | c | c | c | c |}
    \hline 
      & R1 & R2 & R3 & R4 & R5 & R6 & R7 & R8 \\
    \hline
    Menutech         & \ding{52} & \ding{52} & \ding{52} & \ding{56} & \ding{52} & \ding{52} & \ding{52} & \ding{56} \\
    \hline
    Allergy Menu     & \ding{52} & \ding{56} & \ding{56} & \ding{56} & \ding{56} & \ding{56} & \ding{56} & \ding{56} \\
    \hline
    Menu Guide       & \ding{52} & \ding{52} & \ding{56} & \ding{56} & \ding{52} & \ding{52} & \ding{52} & \ding{56} \\
    \hline
    BigZpoon Eagle   & \ding{52} & \ding{56} & \ding{52} & \ding{52} & \ding{52} & \ding{56} & \ding{56} & \ding{56} \\
    \hline
    Allergen Checker & \ding{52} & \ding{56} & \ding{52} & \ding{52} & \ding{52} & \ding{56} & \ding{56} & \ding{56} \\
    \hline
    Choose Well      & \ding{52} & \ding{52} & \ding{52} & \ding{52} & \ding{52} & \ding{52} & \ding{56} & \ding{52} \\
    \hline
  \end{tabular}
  \caption{Comparison of applications regarding use cases with restaurant}\label{comparison-restaurant-use-cases-result}
\end{table}
