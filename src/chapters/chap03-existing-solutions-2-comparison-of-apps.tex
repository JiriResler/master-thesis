\section{Comparison of the existing applications}

\subsection*{Measured use cases}

% - ci aplikacia poskytuje interaktivne menu pre hosta
% - filter only by allergens or also by diets or whole allergies
% - ability for guest to create profile
% - ci je aplikacia na webe alebo stiahnutelna appka
% - moznost pridat do menu QR code

\todo[inline]{add a line break in use case 9 or restructure it, so it is shorter - maybe add an option to display icons instead of allergen names}
\todo[inline]{rozdelit tabulku na dve - v jednej use casy pre hosta v druhej use casy pre restauraciu}
\todo[inline]{Gx: Provide a guest with an interactive menu}
\todo[inline]{Gx: Allow searching for a restaurant}
\begin{center}
  \begin{tabular}{| c | l |}
    \hline
    G1 & Allow a guest to create a dietary profile. \\
    \hline
    G2 & Display a personalized menu to a guest. \\
    \hline
    G3 & Allow a guest to filter out items of a menu which they cannot eat. \\
    \hline
    G4 & Display a menu after a quest scans a QR code on a printed menu. \\
    \hline
    G5 & Allow a guest to mark a restaurant as their favorite. \\
    \hline
    G6 & Allow a guest to control their personal data. \\
    \hline
    G7 & Translate a menu to the language of the UI. \\
    \hline
    G8 & Find restaurants near the guest based on their location. \\
    \hline
    G9 & Allow the guest to pay for a menu item. \\ % aplikácia umožňuje zaplatiť za účet v reštaurácii v rámci aplikácie
    \hline
    G10 & možnosť pracovať s aplikáciou neprihlásený \\
    \hline
  \end{tabular}
  \newline
\end{center}

\todo[inline]{Gx: Application provides pre-defined data (User has to create data)}
\begin{center}
  \begin{tabular}{| c | l |}
    \hline
    R1 & Automatically add allergens to an item of a menu. \\
    \hline
    R2 & Allow a restaurant employee to choose whether to use allergen names or numbers in a menu. \\
    \hline
    R3 & Provide templates for creating a new menu. \\
    \hline
    R4 & Allow a restaurant employee to reuse a previously created menu item. \\
    \hline
    R5 & Allow a restaurant employee to control their restaurant's data. \\    
    \hline
    R6 & weights of meals \\
    \hline 
    R7 & možnosť automaticky vygenerovať menu \\
    \hline
  \end{tabular}
  \newline
\end{center}

\todo[inline]{vertically center contents of the table - may be usable for all tables in the thesis} % https://tex.stackexchange.com/questions/7208/how-to-vertically-center-the-text-of-the-cells/611601#611601
\todo[inline]{make all columns have equal width}
\subsection*{Results of the comparison}
All applications provide their guests with interactive menus which can be suited to the guests' dietary needs.

\begin{center}
  \begin{tabular}{| l | c | c | c | c | c | c | c | c | c | c|}
    \hline 
      & G1 & G2 & G3 & G4 & G5 & G6 & G7 & G8 & G9 & G10 \\
    \hline
    Choose Well & \ding{52} & \ding{52} & \ding{52} & \ding{52} & \ding{52} & \ding{52} & \ding{52} & \ding{52} & \ding{52} & \ding{52} \\
    \hline
    % App2 & \ding{52} & \ding{56} & - \\
    % \hline
  \end{tabular}
  \newline
\end{center}

\begin{center}
  \begin{tabular}{| l | c | c | c | c | c | c | c |}
    \hline 
      & R1 & R2 & R3 & R4 & R5 & R6 & R7 \\
    \hline
    Choose Well & \ding{52} & \ding{52} & \ding{56} & \ding{56} & \ding{52} & \ding{56} & \ding{52} \\
    \hline
    % App2 & \ding{52} & \ding{56} & - \\
    % \hline
  \end{tabular}
  \newline
\end{center}

\ding{52} means that an application fully supports the use case \newline
\ding{56} means that an application does not support the use case \newline
\ding{115} means that an application supports the use case only partially \newline
- means that the use case is skipped as it is out of the application's scope \newline

\todo[inline]{add Other applications worth mentioning section}
% \section{Other applications worth mentioning}
