\subsection{Restaurant use cases}
\todo[inline]{check if all covered reqs are actually covered by the use cases}

\noindent \textbf{x. Use case: Create a new menu}
Covers: R2.8, R2.9, R2.10, R2.12, R2.13, R2.14, R2.15, R2.16
\begin{center}
  \begin{tabular}{| l | p{10.75cm} | }
    \hline
    Actor        & Restaurant \\
    \hline
    Scenario     &
    \begin{minipage}[t]{\linewidth}
      \begin{enumerate}[leftmargin=*,nosep,before=\vspace{-0.575\baselineskip},after=\strut]
        \item The restaurant employee presses a button for creating a new menu.
        \item The application displays a screen with an empty menu. \textbf{A1} \textbf{A2} 
        \item The restaurant employee specifies meta-information about the menu. \textbf{A3}
        \item The restaurant employee specifies the restaurant's contact information. \textbf{A4} \textbf{A5} 
        \todo[inline]{Extension point - all menu options}
        \item The restaurant employee specifies when shall the menu be valid. (Extension point)
        \todo[inline]{Extension point - Warn my guests about allergens in the foods I serve}
        \item The restaurant employee specifies how shall the menu display allergen information. (Extension point) 
        \item The restaurant employee adds categories to the~menu.~\textbf{A6}
        \item The restaurant employee adds individual food items to the menu.
        \item The restaurant employee presses a button for saving the menu.
        \todo[inline]{Extension point - publish, print}
        \item The application saves the menu. \textbf{A7} (Extension point)
      \end{enumerate}
    \end{minipage}
    \\
    \hline
    Alternatives &
    \begin{minipage}[t]{\linewidth}
      \begin{description}[nosep,after=\strut]
        \item [A1:] The restaurant employee selects that they wish to use a template. The application displays a menu with some predefined content.
        \item [A2:] The restaurant employee selects that they wish to create the menu based on an existing menu. The application displays a menu with the contents of the base menu.
        \item [A3:] Meta-information is loaded from the restaurant's profile and inserted to the menu by the application.
        \item [A4:] The restaurant employee does not wish to include the restaurant's contact information in the menu and skips this process. 
        \item [A5:] The application inserts the restaurant's contact information automatically based on the restaurant's profile.
        \item [A6:] The restaurant employee skips the creation of categories.
        \item [A7:] The restaurant employee forgets to specify one of the menu's meta-information options. The application warns the restaurant employee about this fact. The restaurant employee finalizes the creation of the menu and the scenario continues with step 9.
      \end{description}
    \end{minipage}
    \\
    \hline
  \end{tabular}
  \newline
\end{center}

\newpage

\noindent \textbf{x. Use case: Warn my guests about allergens in the foods I serve}
Extends: Create a new menu
Covers: R2.11
\todo[inline]{add use case number in extends column}
\begin{center}
  \begin{tabular}{| l | p{10.75cm} | }
    \hline
    Actor        & Restaurant \\
    \hline
    Description        & A restaurant employee wants to add information about allergens contained in a menu's items. \\
    \hline
    Extends       &  x: Create a new menu \\
    \hline
    Scenario     &
    \begin{minipage}[t]{\linewidth}
      \begin{enumerate}[leftmargin=*,nosep,before=\vspace{-0.575\baselineskip},after=\strut]
        \item The restaurant employee specifies that numbers shall be used to indicate allergens contained in menu items. \textbf{A1}
        \item The application inserts a predefined allergen table at the end of the menu.
        \item The application later automatically inserts allergens to menu items based on their ingredients.
      \end{enumerate}
    \end{minipage}
    \\
    \hline
    Alternatives &
    \begin{minipage}[t]{\linewidth}
      \begin{description}[nosep,after=\strut]
        \item [A1:] The restaurant employee specifies that allergen labels shall be used in the menu. The scenario continues with step 3.
      \end{description}
    \end{minipage}
    \\
    \hline
  \end{tabular}
  \newline
\end{center}

\noindent \textbf{x. Use case: Create a stable menu}
Covers: R2.2
\begin{center}
  \begin{tabular}{| l | p{10.75cm} | }
    \hline
    Actor        & Restaurant \\
    \hline
    Description  & A restaurant's management wants its restaurant to have a stable menu which will be valid every day of the week. \\
    \hline
    Extends       &  x: Create a new menu \\
    \hline
    Scenario     &
    \begin{minipage}[t]{\linewidth}
      \begin{enumerate}[leftmargin=*,nosep,before=\vspace{-0.575\baselineskip},after=\strut]
        \item The restaurant employee selects that the menu shall be valid every day of the week. \textbf{A1} 
        \item The restaurant employee chooses an option that the menu shall repeat periodically for the selected days.
        \item The restaurant employee specifies that the menu shall be valid all day.
      \end{enumerate}
    \end{minipage}
    \\
    \hline
    Alternatives &
    \begin{minipage}[t]{\linewidth}
      \begin{description}[nosep,after=\strut]
        \item [A1:] The restaurant employee selects the option that the menu will be a stable offer menu. The application inserts the information about what days of the week shall the menu be valid and when automatically.
      \end{description}
    \end{minipage}
    \\
    \hline
  \end{tabular}
  \newline
\end{center}

\noindent \textbf{x. Use case: Create a list of beverages}
Covers: R2.3
\begin{center}
  \begin{tabular}{| l | p{10.75cm} | }
    \hline
    Actor        & Restaurant \\
    \hline
    Description  &  \\
    \hline
    Extends       &  x: Create a new menu \\
    \hline
    Scenario     &
    \begin{minipage}[t]{\linewidth}
      \begin{enumerate}[leftmargin=*,nosep,before=\vspace{-0.575\baselineskip},after=\strut]
        \item The restaurant employee selects that the menu shall be valid every day of the week. \textbf{A1}
        \item The restaurant employee chooses an option that the menu shall repeat periodically for the selected days.
        \item The restaurant employee specifies that the menu shall be valid all day.
        \item The restaurant employee later adds beverages as items of the menu.
      \end{enumerate}
    \end{minipage}
    \\
    \hline
    Alternatives &
    \begin{minipage}[t]{\linewidth}
      \begin{description}[nosep,after=\strut]
        \item [A1:] The restaurant employee selects the option that the menu will be a stable offer of beverages. The application inserts the information about what days of the week shall the menu be valid and when automatically. The restaurant employee continues with step 4.
      \end{description}
    \end{minipage}
    \\
    \hline
  \end{tabular}
  \newline
\end{center}

\noindent \textbf{x. Use case: Create a daily menu for tomorrow}
Covers: R2.1
\begin{center}
  \begin{tabular}{| l | p{10.75cm} | }
    \hline
    Actor        & Restaurant \\
    \hline
    Extends       &  x: Create a new menu \\
    \hline
    Scenario     &
    \begin{minipage}[t]{\linewidth}
      \begin{enumerate}[leftmargin=*,nosep,before=\vspace{-0.575\baselineskip},after=\strut]
        \item The restaurant employee specifies the day for which will the menu be valid.
        \item The restaurant employee specifies the time range for when will the menu be served.
      \end{enumerate}
    \end{minipage}
    \\
    \hline
  \end{tabular}
  \newline
\end{center}

\noindent \textbf{x. Use case: Create daily menus for the next week}
Covers: R2.1
\begin{center}
  \begin{tabular}{| l | p{10.75cm} | }
    \hline
    Actor        & Restaurant \\
    \hline
    Scenario     &
    \begin{minipage}[t]{\linewidth}
      \begin{enumerate}[leftmargin=*,nosep,before=\vspace{-0.575\baselineskip},after=\strut]
        \item The restaurant employee specifies the day for which will the menu be valid.
        \item The restaurant employee continues the process of creating the menu.
        \item When the menu is finished, creates the restaurant employee another menu. \textbf{A1}
      \end{enumerate}
    \end{minipage}
    \\
    \hline
    Alternatives &
    \begin{minipage}[t]{\linewidth}
      \begin{description}[nosep,after=\strut]
        \item [A1:] The restaurant employee has created menus for the whole week. The scenario ends.
      \end{description}
    \end{minipage}
    \\
    \hline
  \end{tabular}
  \newline
\end{center}

\noindent \textbf{x. Use case: Create a daily menu which will repeat each Tuesday}
Covers: R2.7
\begin{center}
  \begin{tabular}{| l | p{10.75cm} | }
    \hline
    Actor        & Restaurant \\
    \hline
    Extends       &  x: Create a new menu \\
    \hline
    Scenario     &
    \begin{minipage}[t]{\linewidth}
      \begin{enumerate}[leftmargin=*,nosep,before=\vspace{-0.575\baselineskip},after=\strut]
        \item The restaurant employee specifies that the menu will be valid on Tuesday.
        \item The restaurant employee checks the option that the menu will repeat periodically.
        \item The restaurant employee specifies the time range for when will the menu be served.
      \end{enumerate}
    \end{minipage}
    \\
    \hline
  \end{tabular}
  \newline
\end{center}

\noindent \textbf{x. Share a menu on social media}
Covers: R2.17
\begin{center}
  \begin{tabular}{| l | p{10.75cm} | }
    \hline
    Actor        & Restaurant \\
    \hline
    Scenario     &
    \begin{minipage}[t]{\linewidth}
      \begin{enumerate}[leftmargin=*,nosep,before=\vspace{-0.575\baselineskip},after=\strut]
        \item The restaurant employee clicks a button for sharing the menu.
        \item The application generates a URL which points to the menu.
        \item The restaurant employee shares the URL on the restaurant's social media. \textbf{A1}
      \end{enumerate}
    \end{minipage}
    \\
    \hline
  \end{tabular}
  \newline
\end{center}

\noindent \textbf{x. Use case: Change an ingredient of a meal in an existing menu}
Covers: R2.5
\begin{center}
  \begin{tabular}{| l | p{10.75cm} | }
    \hline
    Actor        & Restaurant \\
    \hline
    Scenario     &
    \begin{minipage}[t]{\linewidth}
      \begin{enumerate}[leftmargin=*,nosep,before=\vspace{-0.575\baselineskip},after=\strut]
        \item The restaurant employee clicks an "Edit" button next to a menu.
        \item The application lets the restaurant employee edit the menu.
        \item The restaurant employee clicks a "Save" button.
        \item The application saves the menu.
      \end{enumerate}
    \end{minipage}
    \\
    \hline
  \end{tabular}
  \newline
\end{center}

\noindent \textbf{x. Delete a menu}
Covers: R2.6
\begin{center}
  \begin{tabular}{| l | p{10.75cm} | }
    \hline
    Actor        & Restaurant \\
    \hline
    Scenario     &
    \begin{minipage}[t]{\linewidth}
      \begin{enumerate}[leftmargin=*,nosep,before=\vspace{-0.575\baselineskip},after=\strut]
        \item The restaurant employee clicks a button for sharing the menu.
        \item The application generates a URL which points to the menu.
        \item The restaurant employee shares the URL on the restaurant's social media. \textbf{A1}
      \end{enumerate}
    \end{minipage}
    \\
    \hline
  \end{tabular}
  \newline
\end{center}

\todo[inline]{change name in figure}
\noindent \textbf{x. Use case: Print a QR code for a menu}
Covers: R2.4
\begin{center}
  \begin{tabular}{| l | p{10.75cm} | }
    \hline
    Actor        & Restaurant \\
    \hline
    Description  & A restaurant employee wishes to print a menu and place it on tables. \\
    \hline
    Scenario     &
    \begin{minipage}[t]{\linewidth}
      \begin{enumerate}[leftmargin=*,nosep,before=\vspace{-0.575\baselineskip},after=\strut]
        \item The restaurant employee clicks the "Print" button associated with an existing menu.
        \item The application enables the restaurant employee to print the menu.
      \end{enumerate}
    \end{minipage}
    \\
    \hline
  \end{tabular}
  \newline
\end{center}

\noindent \textbf{x. Use case: Have control over my data}
Covers: 3.7
\begin{center}
  \begin{tabular}{| l | p{10.75cm} | }
    \hline
    Actor        & Restaurant \\
    \hline
    Description  & A restaurant employee wants to specify where should the application store and read the restaurant's data. \\
    \hline
    Scenario     &
    \begin{minipage}[t]{\linewidth}
      \begin{enumerate}[leftmargin=*,nosep,before=\vspace{-0.575\baselineskip},after=\strut]
        \item The restaurant employee navigates to a page for managing data storage options.
        \item The application provides a list of places where it can store data.
        \item The restaurant employee selects one of the options.
        \item The application starts using the selected place for storing and reading the restaurant's data.
      \end{enumerate}
    \end{minipage}
    \\
    \hline
  \end{tabular}
  \newline
\end{center}

\begin{figure}[h]
  \centering
  \includegraphics[width=0.62\linewidth]{master-thesis/img/use-cases/use_cases_restaurant_publish_menu}
  \caption{Restaurant offer use cases}
\end{figure}

\begin{figure}[h]
  \centering
  \includegraphics[width=0.62\linewidth]{master-thesis/img/use-cases/use_cases_restaurant_create_menu}
  \caption{Restaurant menu creation use cases}
\end{figure}

\begin{figure}[h]
  \centering
  \includegraphics[width=0.62\linewidth]{master-thesis/img/use-cases/use_cases_restaurant_create_daily_menu}
  \caption{Restaurant daily menu creation use cases}
\end{figure}
