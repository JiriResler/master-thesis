\todo[inline]{Add use case coverage of requirements}
\section{Use case scenarios}
Now we are going to depict goals of users which the application will make achievable.
Figure 2.1 contains a use case diagram of the application.
A use case styled with a bold border groups together more use cases and will be further expanded.
All use cases will be structurally described by independent use case scenarios.
A scenario will contain its description only if the goal of a user is not clear from a use case's name.

\begin{figure}[h]
  \includegraphics[width=\linewidth]{master-thesis/img/use_cases/use_cases}
  \caption{The application's use case diagram.}
\end{figure}

\newpage

% Add top padding for scenario tables
\def\arraystretch{1.5}

\subsection{Guest use cases}

\begin{figure}[h]
  \centering
  \includegraphics[width=0.62\linewidth]{master-thesis/img/use_cases/use_cases_guest_profile_management}
  \caption{Guest profile management use cases}
\end{figure}

\newpage

\noindent \textbf{1. Use case: Specify what allergies I have}
\begin{center}
  \begin{tabular}{| l | p{10.75cm} | }
    \hline
    Actor       & Guest \\
    \hline
    Scenario    &
    \begin{minipage}[t]{\linewidth}
      \begin{enumerate}[leftmargin=*,nosep,before=\vspace{-0.575\baselineskip},after=\strut]
        \item The guest opens their profile and navigates to a section for managing allergies.
        \item The application displays a screen with a list of previously specified allergies by the guest. \textbf{A1}
        \item The guest presses a button labeled as "Add allergy".
        \item The application displays a search bar.
        \item The guest starts typing the name of an allergy into the search bar.
        \item The application suggests allergies which contain the given input in their name.
        \item The guest selects the desired allergy and presses an "Add" button. \textbf{A2}
        \item The application adds the specified allergy the guest's profile. \textbf{A3}
        \item The guest repeats steps 3 to 8 until they have specified all of the food allergies they suffer from.
      \end{enumerate}
    \end{minipage}
    \\
    \hline
    Alternatives &
    \begin{minipage}[t]{\linewidth}
      \begin{description}[nosep,after=\strut]
        \item [A1:] The list is empty because the guest has not specified any allergies yet. The application displays a text containing this information.
        \item [A2:] The application does not recognize the allergy which the guest is trying to specify. The guest creates a public issue in the application's repository with a request to add the desired allergy to the application.
        \item [A3:] The allergy the guest has specified is already contained in the guest's profile. The application informs the guest about this fact and their profile is not altered.
      \end{description}
    \end{minipage}
    \\
    \hline
  \end{tabular}
  \newline
\end{center}

\newpage

\noindent \textbf{2. Use case: Specify what diets I am on}
\begin{center}
  \begin{tabular}{| l | p{10.75cm} | }
    \hline
    Actor       & Guest \\
    \hline
    Scenario    &
    \begin{minipage}[t]{\linewidth}
      \begin{enumerate}[leftmargin=*,nosep,before=\vspace{-0.575\baselineskip},after=\strut]
        \item The guest opens their profile and navigates to a section for managing diets.
        \item The application displays a screen with a list of previously specified diets by the guest. \textbf{A1}
        \item The guest presses a button labeled as "Add diet".
        \item The application displays a search bar.
        \item The guest starts typing the name of a diet into the search bar.
        \item The application suggests diets which contain the given input in their name.
        \item The guest selects the desired diet and presses an "Add" button. \textbf{A2}
        \item The application adds the specified diet to the guest's profile. \textbf{A3}
        \item The guest repeats steps 3 to 8 until they have specified all of the diets they are on.
      \end{enumerate}
    \end{minipage}
    \\
    \hline
    Alternatives &
    \begin{minipage}[t]{\linewidth}
      \begin{description}[nosep,after=\strut]
        \item [A1:] The list is empty because the guest has not specified any diets yet. The application displays a text containing this information.
        \item [A2:] The application does not recognize the diet which the guest is trying to specify. The guest creates a public issue in the application's repository with a request to add the desired diet to the application.
        \item [A3:] The diet the guest has specified is already contained in the guest's profile. The application informs the guest about this fact and their profile is not altered.
      \end{description}
    \end{minipage}
    \\
    \hline
  \end{tabular}
  \newline
\end{center}

\newpage

\noindent \textbf{3. Use case: Specify what food and beverages I like and dislike}
\begin{center}
  \begin{tabular}{| l | p{10.75cm} | }
    \hline
    Actor    & Guest \\
    \hline
    Scenario &
    \begin{minipage}[t]{\linewidth}
      \begin{enumerate}[leftmargin=*,nosep,before=\vspace{-0.575\baselineskip},after=\strut]
        \item The guest opens their profile and navigates to a section for managing food preferences.
        \item The application displays a screen with two lists, one containing the foods which the guest likes and the other containing the foods which the guest dislikes. \textbf{A1}
        \item The guest presses a button labeled as "Add food" at the end of the list of foods which they like. \textbf{A2}
        \item The application displays a search bar.
        \item The guest starts typing the name of a food into the search bar.
        \item The application suggests foods which contain the given input text in their name.
        \item The guest selects the desired food and presses an "Add" button. \textbf{A3}
        \item The application adds the specified food to the guest's profile. \textbf{A4}
        \item The guest repeats steps 3 to 8 until they have specified all of their food preferences.
      \end{enumerate}
    \end{minipage}
    \\
    \hline
    Alternatives &
    \begin{minipage}[t]{\linewidth}
      \begin{description}[nosep,after=\strut]
        \item [A1:] Either one or both of the lists are empty because the guest has not specified any of their preferences yet. The application displays a text which informs the guest about this fact.
        \item [A2:] The guest presses a button labeled as "Add food" at the end of the list of foods which they dislike.
        \item [A3:] The application does not recognize the food which the guest is trying to specify. The guest creates a public issue in the application's repository with a request to add the desired food to the application.
        \item [A4:] The guest's profile already contains the specified food. The guest is informed about this fact and their profile is not altered.
      \end{description}
    \end{minipage}
    \\
    \hline
  \end{tabular}
  \newline
\end{center}

\newpage

\noindent \textbf{4. Use case: Manage a list of my favorite restaurants}

\begin{center}
  \begin{tabular}{| l | p{10.75cm} | }
    \hline
    Actor    & Guest \\
    \hline
    Scenario &
    \begin{minipage}[t]{\linewidth}
      \begin{enumerate}[leftmargin=*,nosep,before=\vspace{-0.575\baselineskip},after=\strut]
        \item The guest opens their profile and navigates to a section where they can manage their favorite restaurants. \textbf{A1}
        \item The application displays a screen with a list containing the guest's favorite restaurants. \textbf{A2} 
        \item The guest presses a button for adding a restaurant to~the~list.~\textbf{A3}
        \item The application provides a text input field. 
        \item The guest specifies the IRI of a restaurant which they wish to add to the list and presses an "Add" button. 
        \item The application adds the restaurant to the~guest's~profile.~\textbf{A4}~\textbf{A5}
      \end{enumerate}
    \end{minipage}
    \\
    \hline
    Alternatives &
    \begin{minipage}[t]{\linewidth}
      \begin{description}[nosep,after=\strut]
        \item [A1:] The guest is viewing a restaurant's detail. The guest and presses a button for adding the restaurant to their profile. The application adds the restaurant to the guest's profile. \textbf{A1.b}
        \item [A1.b:] The restaurant is already contained in the guest's profile and pressing the button removes the restaurant from it.
        \item [A2:] The list is empty because the guest has not added any restaurants yet. The application displays a text containing this information.
        \item [A3:] The guest presses a button for removing a restaurant. The application removes the restaurant from the guest's profile.
        \item [A4:] The guest's profile already contains the restaurant by the specified IRI. The guest is informed about this fact and their profile is not altered.
        \item [A5:] The IRI which the guest specified is not valid. The application informs the guest about this fact and the scenario continues with step 3.
      \end{description}
    \end{minipage}
    \\
    \hline
  \end{tabular}
  \newline
\end{center}

\newpage

\todo[inline]{Make the figure bigger - increase height, make bubbles bigger}

\begin{figure}[h]
  \centering
  \includegraphics[width=0.62\linewidth]{master-thesis/img/use_cases/use_cases_guest_menu_viewer}
  \caption{Guest menu viewing use cases}
\end{figure}

\textbf{5. Use case: Find a meal I can eat at a restaurant}

\begin{center}
  \begin{tabular}{| l | p{10.75cm} | }
    \hline
    Actor        & Guest \\
    \hline
    Description  & A guest comes to a restaurant and is deciding what to order. \\
    \hline
    Scenario     &
    \begin{minipage}[t]{\linewidth}
      \begin{enumerate}[leftmargin=*,nosep,before=\vspace{-0.575\baselineskip},after=\strut]
        \item The guest scans a QR code on a printed menu which takes them to the application.
        \item The application loads and displays an online version of the printed menu.
        \item The guest selects that they wish to filter out meals of the menu which do not correspond to their profile. \textbf{A1}
        \item The application restructures the displayed menu.
      \end{enumerate}
    \end{minipage}
    \\
    \hline
    Alternatives &
    \begin{minipage}[t]{\linewidth}
      \begin{description}[nosep,after=\strut]
        \item [A1:] The guest selects that they wish to sort the menu by what menu items the guest can eat according to their~profile.
      \end{description}
    \end{minipage}
    \\
    \hline
  \end{tabular}
  \newline
\end{center}

\newpage

\noindent \textbf{6. Use case: Look up online what a restaurant offers today}

\begin{center}
  \begin{tabular}{| l | p{10.75cm} | }
    \hline
    Actor        & Guest \\
    \hline
    Description  & A guest is at home or on their way to a restaurant and wants to know what the restaurant serves at the moment. \\
    \hline
    Scenario     &
    \begin{minipage}[t]{\linewidth}
      \begin{enumerate}[leftmargin=*,nosep,before=\vspace{-0.575\baselineskip},after=\strut]
        \item The guest provides the IRI of the restaurant.
        \item The application displays a detail of the restaurant. \textbf{A1}
        \item The guest selects a menu.
        \item The application displays the menu.
      \end{enumerate}
    \end{minipage}
    \\
    \hline
    Alternatives &
    \begin{minipage}[t]{\linewidth}
      \begin{description}[nosep,after=\strut] 
        \item [A1:] The specified IRI is not valid. The application informs the guest about this fact and the scenario starts over from step 1.
      \end{description}
    \end{minipage}
    \\
    \hline
  \end{tabular}
  \newline
\end{center}

% \noindent \textbf{4. Use case: Find currently served meals by my favorite restaurants which correspond to my profile}

% \begin{center}
%   \begin{tabular}{| l | p{10.75cm} | }
%     \hline
%     Actor        & Guest \\
%     \hline
%     Description  & A guest wants to see what do their favorite restaurants currently have to offer. \\
%     \hline
%     Scenario     &
%     \begin{minipage}[t]{\linewidth}
%       \begin{enumerate}[leftmargin=*,nosep,before=\vspace{-0.575\baselineskip},after=\strut]
%         \item The application provides an overview of the guest's favorite restaurants.
%         \item The application shows currently valid menus of the guest's favorite restaurants by default. A1 A2 - currently no meals are served A3 meals are served but none correspond to profile
%       \end{enumerate}
%     \end{minipage}
%     \\
%     \hline
%     Alternatives &
%     \begin{minipage}[t]{\linewidth}
%       \begin{description}[nosep,after=\strut]
%         \item [A1:] The guest has no restaurants in their list of favorite restaurants. The application displays a screen with instructions on how to add a restaurant to the list.
%         \item [A2:] The restaurants in the guest's list of favorite restaurants have no menus uploaded which are valid at the moment. The application displays a screen which says 

%         zobrazit restauracie
%         zobrazit jedla
      
%       \end{description}
%     \end{minipage}
%     \\
%     \hline
%   \end{tabular}
%   \newline
% \end{center}

% \todo[inline]{What does it precisely mean to have control over one's data? Where is data stored, who has access to it}
% \textbf{x. Use case: Have control over my data}

% \begin{center}
%   \begin{tabular}{| l | p{10.75cm} | }
%     \hline
%     Actor        & Guest \\
%     \hline
%     Description  & A guest wishes to specify where should the application store their data. \\
%     \hline
%     Scenario     &
%     \begin{minipage}[t]{\linewidth}
%       \begin{enumerate}[leftmargin=*,nosep,before=\vspace{-0.575\baselineskip},after=\strut]
%         \item  \textbf{A1}
%         \item 
%         \item 
%         \item 
%         \item 
%       \end{enumerate}
%     \end{minipage}
%     \\
%     \hline
%     Alternatives &
%     \begin{minipage}[t]{\linewidth}
%       \begin{description}[nosep,after=\strut]
%         \item [A1:] 
%       \end{description}
%     \end{minipage}
%     \\
%     \hline
%   \end{tabular}
%   \newline
% \end{center}

\subsection{Restaurant use cases}

\begin{figure}[h]
  \centering
  \includegraphics[width=0.62\linewidth]{master-thesis/img/use_cases/use_cases_restaurant_publish_menu}
  \caption{Restaurant offer use cases}
\end{figure}

\begin{figure}[h]
  \centering
  \includegraphics[width=0.62\linewidth]{master-thesis/img/use_cases/use_cases_restaurant_create_menu}
  \caption{Restaurant menu creation use cases}
\end{figure}

\begin{figure}[h]
  \centering
  \includegraphics[width=0.62\linewidth]{master-thesis/img/use_cases/use_cases_restaurant_create_daily_menu}
  \caption{Restaurant daily menu creation use cases}
\end{figure}

% \newpage

% \subsection{Restaurant use cases}

% \noindent \textbf{8. Use case: Post a menu online}

% \begin{center}
%   \begin{tabular}{| l | p{10.75cm} | }
%     \hline
%     Actor        & Restaurant \\
%     \hline
%     Description  & A restaurant's management decides to post their currently valid menu online. \\
%     \hline
%     Scenario     &
%     \begin{minipage}[t]{\linewidth}
%       \begin{enumerate}[leftmargin=*,nosep,before=\vspace{-0.575\baselineskip},after=\strut]
%         \item The restaurant employee creates the menu as in the use case x.
%         \item The application generates a URL which points to the application, providing it with the created menu.
%         \item The restaurant employee adds the generated URL to the restaurant's webpage. \textbf{A1}
%       \end{enumerate}
%     \end{minipage}
%     \\
%     \hline
%     Alternatives &
%     \begin{minipage}[t]{\linewidth}
%       \begin{description}[nosep,after=\strut]
%         \item [A1:] The restaurant employee shares the generated URL on the restaurant's social media.
%       \end{description}
%     \end{minipage}
%     \\
%     \hline
%   \end{tabular}
%   \newline
% \end{center}

% \noindent \textbf{9. Use case: Allow my guests to view a menu by scanning a QR code}

% \begin{center}
%   \begin{tabular}{| l | p{10.75cm} |}
%     \hline
%     Actor        & Restaurant \\
%     \hline
%     Description  & A restaurant's management would like to provide a menu with a QR code which will take the restaurant's guests to the application. \\
%     \hline
%     Scenario     &
%     \begin{minipage}[t]{\linewidth}
%       \begin{enumerate}[leftmargin=*,nosep,before=\vspace{-0.575\baselineskip},after=\strut]
%         \item The restaurant employee creates a menu as in use case x, ensuring that a checkbox labeled "Add a QR code" is checked. \textbf{A1}
%         \item The application generates a QR code which will link a guest to the application.
%         \item The restaurant employee prints the menu and puts in on tables as in use case x.
%         \item A guest scans the QR code on the printed menu.
%         \item The application displays the menu.
%       \end{enumerate}
%     \end{minipage}
%     \\
%     \hline
%     Alternatives &
%     \begin{minipage}[t]{\linewidth}
%       \begin{description}[nosep,after=\strut]
%         \item [A1:] The restaurant employee selects a menu from previously created menus.
%       \end{description}
%     \end{minipage}
%     \\
%     \hline
%   \end{tabular}
%   \newline
% \end{center}

% \noindent \textbf{10. Use case: Change an ingredient of a meal in an existing menu}

% \begin{center}
%   \begin{tabular}{| l | p{10.75cm} | }
%     \hline
%     Actor        & Restaurant \\
%     \hline
%     Scenario     &
%     \begin{minipage}[t]{\linewidth}
%       \begin{enumerate}[leftmargin=*,nosep,before=\vspace{-0.575\baselineskip},after=\strut]
%         \item The restaurant employee logs in to the application.
%         \item The restaurant employee clicks an "Edit" button next to a menu.
%         \item The application lets the restaurant employee edit the menu.
%       \end{enumerate}
%     \end{minipage}
%     \\
%     % \hline
%     % Alternatives &
%     % \begin{minipage}[t]{\linewidth}
%     %   \begin{description}[nosep,after=\strut]
%     %     \item [A1:] ...
%     %   \end{description}
%     % \end{minipage}
%     % \\
%     \hline
%   \end{tabular}
%   \newline
% \end{center}

% \noindent \textbf{11. Use case: Place a menu on tables}

% \begin{center}
%   \begin{tabular}{| l | p{10.75cm} | }
%     \hline
%     Actor        & Restaurant \\
%     \hline
%     Description  & A restaurant employee would like to print a menu and place it on tables. \\
%     \hline
%     Scenario     &
%     \begin{minipage}[t]{\linewidth}
%       \begin{enumerate}[leftmargin=*,nosep,before=\vspace{-0.575\baselineskip},after=\strut]
%         \item The restaurant employee logs in to the application.
%         \item The restaurant employee creates a new menu. \textbf{A1}
%         \item The application displays a detail of the menu.
%         \item The restaurant employee clicks a "Print" button in the detail of the menu.
%         \item The application manages to print the menu.
%       \end{enumerate}
%     \end{minipage}
%     \\
%     \hline
%     Alternatives &
%     \begin{minipage}[t]{\linewidth}
%       \begin{description}[nosep,after=\strut]
%         \item [A1:] The restaurant employee selects an existing menu and proceeds with step 3.
%       \end{description}
%     \end{minipage}
%     \\
%     \hline
%   \end{tabular}
%   \newline
% \end{center}

% \noindent \textbf{13. Use case: Reuse an existing daily menu for today.}

% \begin{center}
%   \begin{tabular}{| l | p{10.75cm} | }
%     \hline
%     Actor        & Restaurant \\
%     \hline
%     Description  &  \\
%     \hline
%     Scenario     &
%     \begin{minipage}[t]{\linewidth}
%       \begin{enumerate}[leftmargin=*,nosep,before=\vspace{-0.575\baselineskip},after=\strut]
%         \item The restaurant employee logs in to the application.
%         \item The restaurant employee clicks a button labeled "Create new menu". \textbf{A1}
%         \item The application asks the restaurant employee what kind of menu they would like to create.
%         \item The restaurant employee selects that they wish to create a daily menu.
%         \item The application asks the restaurant employee whether they want to use an existing menu as a template.
%         \item The restaurant employee confirms by clicking a "Yes" button.
%         \item The application displays a list of menus as possible templates.
%         \item The restaurant employee chooses a menu from the list.
%         \item The application loads the chosen menu.
%         \item The restaurant employee changes the date of validity of the menu.
%         \item The restaurant employee clicks a "Save menu" button.
%         \item The application asks the restaurant employee whether they would like to overwrite the existing menu.
%         \item The restaurant employee denies by clicking a "No, create a new menu" button. \textbf{A2}
%         \item The application saves the menu.
%       \end{enumerate}
%     \end{minipage}
%     \\
%     \hline
%     Alternatives &
%     \begin{minipage}[t]{\linewidth}
%       \begin{description}[nosep,after=\strut]
%         \item [A1:] The restaurant employee clicks an "Edit" button next to a menu which they want to reuse. The restaurant employee edits a date field which controls for which day the menu is valid. The restaurant employee then clicks a "Save menu" button and the application saves the menu.
%         \item [A2:] The restaurant employee confirms by clicking a "Yes" button and the application overwrites the old menu with the new one.
%       \end{description}
%     \end{minipage}
%     \\
%     \hline
%   \end{tabular}
%   \newline
% \end{center}

% \begin{figure}[h]
%   \centering
%   \includegraphics[width=0.62\linewidth]{master-thesis/img/use_cases_restaurant_menu_management}
%   \caption{Restaurant menu creation use cases}
% \end{figure}

% \noindent \textbf{14. Use case: Specify allergens contained in an item of a menu}

% \begin{center}
%   \begin{tabular}{| l | p{10.75cm} | }
%     \hline
%     Actor        & Restaurant \\
%     \hline
%     Description  &  \\
%     \hline
%     Scenario     &
%     \begin{minipage}[t]{\linewidth}
%       \begin{enumerate}[leftmargin=*,nosep,before=\vspace{-0.575\baselineskip},after=\strut]
%         \item ...
%         \item ... \textbf{A1}
%         \item ...
%       \end{enumerate}
%     \end{minipage}
%     \\
%     \hline
%     Alternatives &
%     \begin{minipage}[t]{\linewidth}
%       \begin{description}[nosep,after=\strut]
%         \item [A1:] ...
%       \end{description}
%     \end{minipage}
%     \\
%     \hline
%   \end{tabular}
%   \newline
% \end{center}

% \noindent \textbf{15. Use case: Specify items of a menu}

% \begin{center}
%   \begin{tabular}{| l | p{10.75cm} | }
%     \hline
%     Actor        & Restaurant \\
%     \hline
%     Description  &  \\
%     \hline
%     Scenario     &
%     \begin{minipage}[t]{\linewidth}
%       \begin{enumerate}[leftmargin=*,nosep,before=\vspace{-0.575\baselineskip},after=\strut]
%         \item ...
%         \item ... \textbf{A1}
%         \item ...
%       \end{enumerate}
%     \end{minipage}
%     \\
%     \hline
%     Alternatives &
%     \begin{minipage}[t]{\linewidth}
%       \begin{description}[nosep,after=\strut]
%         \item [A1:] ...
%       \end{description}
%     \end{minipage}
%     \\
%     \hline
%   \end{tabular}
%   \newline
% \end{center}

% \noindent \textbf{16. Use case: Create a daily menu for tomorrow}

% \begin{center}
%   \begin{tabular}{| l | p{10.75cm} | }
%     \hline
%     Actor        & Restaurant \\
%     \hline
%     Scenario     &
%     \begin{minipage}[t]{\linewidth}
%       \begin{enumerate}[leftmargin=*,nosep,before=\vspace{-0.575\baselineskip},after=\strut]
%         \item The restaurant employee logs in to the application.
%         \item The restaurant employee clicks a button labeled "Create new menu".
%         \item The application asks the restaurant employee what kind of menu they would like to create.
%         \item The restaurant employee chooses that they wish to create a daily menu.
%         \item The application asks the restaurant employee whether they want to use an existing menu as a template.
%         \item The restaurant employee denies by clicking a "No, create a new menu" button.
%         \item The restaurant employee specifies items of the menu.
%         \item The restaurant employee specifies which day the menu will be valid.
%         \item The restaurant employee clicks a "Save menu" button.
%         \item The application saves the menu.
%       \end{enumerate}
%     \end{minipage}
%     \\
%     \hline
%     % Alternatives &
%     % \begin{minipage}[t]{\linewidth}
%     %   \begin{description}[nosep,after=\strut]
%     %     \item [A1:] ...
%     %   \end{description}
%     % \end{minipage}
%     % \\
%     % \hline
%   \end{tabular}
%   \newline
% \end{center}

% \noindent \textbf{17. Use case: Specify allergens contained in a meal}

% \begin{center}
%   \begin{tabular}{| l | p{10.75cm} | }
%     \hline
%     Actor        & Restaurant \\
%     \hline
%     Scenario     &
%     \begin{minipage}[t]{\linewidth}
%       \begin{enumerate}[leftmargin=*,nosep,before=\vspace{-0.575\baselineskip},after=\strut]
%         \item The restaurant employee logs in to the application.
%         \item The restaurant employee clicks a "Create a menu" button.
%         \item The application asks the restaurant employee what kind of menu they would like to create.
%         \item The restaurant employee selects one of the options.
%         \item The restaurant employee clicks a plus sign on the new menu to add an item to it.
%         \item The restaurant employee adds information about the food item.
%         \item The application gives the restaurant employee a list of all allergens. \textbf{A1}
%         \item The restaurant employee selects allergens which are contained in the meal.
%         \item The restaurant employee clicks a "Save menu" button.
%         \item The application saves the menu.
%       \end{enumerate}
%     \end{minipage}
%     \\
%     \hline
%     Alternatives &
%     \begin{minipage}[t]{\linewidth}
%       \begin{description}[nosep,after=\strut]
%         \item [A1:] The restaurant employee starts typing the name of an ingredient. The application provides predefined ingredients and automatically adds allergens contained in the selected ingredient as the food item's allergens. The restaurant employee then continues with step 9.
%       \end{description}
%     \end{minipage}
%     \\
%     \hline
%   \end{tabular}
%   \newline
% \end{center}

% \noindent \textbf{18. Use case: Create a daily menu which will repeat each Tuesday}

% \begin{center}
%   \begin{tabular}{| l | p{10.75cm} | }
%     \hline
%     Actor        & Restaurant \\
%     \hline
%     Description  &  \\
%     \hline
%     Scenario     &
%     \begin{minipage}[t]{\linewidth}
%       \begin{enumerate}[leftmargin=*,nosep,before=\vspace{-0.575\baselineskip},after=\strut]
%         \item The restaurant employee logs in to the application.
%         \item The restaurant employee clicks a button labeled "Create new menu". 
%         \item The application asks the restaurant employee what kind of menu they would like to create.
%         \item The restaurant employee chooses that they wish to create a daily menu.
%         \item The application asks the restaurant employee whether they want to use an existing menu as a template.
%         \item The restaurant employee denies by clicking a "No, create a new menu" button.
%         \item The restaurant employee specifies items of the menu.
%         \item The restaurant employee specifies that the menu will be valid the next Tuesday.
%         \item The restaurant employee ensures that a checkbox labeled "Repeat every week" is checked.
%         \item The restaurant employee clicks a "Save menu" button.
%         \item The application saves the menu.
%       \end{enumerate}
%     \end{minipage}
%     \\
%     \hline
%     % Alternatives &
%     % \begin{minipage}[t]{\linewidth}
%     %   \begin{description}[nosep,after=\strut]
%     %     \item [A1:] 
%     %   \end{description}
%     % \end{minipage}
%     % \\
%     % \hline
%   \end{tabular}
%   \newline
% \end{center}

% \noindent \textbf{19. Use case: Create a stable menu}

% \begin{center}
%   \begin{tabular}{| l | p{10.75cm} | }
%     \hline
%     Actor        & Restaurant \\
%     \hline
%     Description  & A restaurant's management wants its restaurant to have a stable menu which will be valid every day. \\
%     \hline
%     Scenario     &
%     \begin{minipage}[t]{\linewidth}
%       \begin{enumerate}[leftmargin=*,nosep,before=\vspace{-0.575\baselineskip},after=\strut]
%         \item The restaurant employee logs in to the application.
%         \item The restaurant employee clicks a button labeled "Create new menu". 
%         \item The application asks the restaurant employee what kind of menu they would like to create.
%         \item The restaurant employee chooses that they wish to create a stable menu.
%         \item The restaurant employee specifies items of the menu.
%         \item The restaurant employee clicks a "Save menu" button.
%         \item The application saves the menu.
%       \end{enumerate}
%     \end{minipage}
%     \\
%     \hline
%     % Alternatives &
%     % \begin{minipage}[t]{\linewidth}
%     %   \begin{description}[nosep,after=\strut]
%     %     \item [A1:] ...
%     %   \end{description}
%     % \end{minipage}
%     % \\
%     % \hline
%   \end{tabular}
%   \newline
% \end{center}

% \noindent \textbf{20. Use case: Create daily menus for the next week}

% \begin{center}
%   \begin{tabular}{| l | p{10.75cm} | }
%     \hline
%     Actor        & Restaurant \\
%     \hline
%     Description  &  \\
%     \hline
%     Scenario     &
%     \begin{minipage}[t]{\linewidth}
%       \begin{enumerate}[leftmargin=*,nosep,before=\vspace{-0.575\baselineskip},after=\strut]
%         \item The restaurant employee logs in to the application.
%         \item The restaurant employee clicks a button labeled "Create new menu".
%         \item The application asks the restaurant employee what kind of menu they would like to create.
%         \item The restaurant employee chooses that they wish to create a daily menu.
%         \item The restaurant employee specifies items of the menu.
%         \item The restaurant employee specifies which day the menu will be valid.
%         \item The restaurant employee clicks a "Save menu" button.
%         \item The application saves the menu.
%         \item The restaurant employee begins again with step 2 until menus for the whole week are created.
%       \end{enumerate}
%     \end{minipage}
%     \\
%     \hline
%     % Alternatives &
%     % \begin{minipage}[t]{\linewidth}
%     %   \begin{description}[nosep,after=\strut]
%     %     \item [A1:] ...
%     %   \end{description}
%     % \end{minipage}
%     % \\
%     % \hline
%   \end{tabular}
%   \newline
% \end{center}

% \noindent \textbf{21. Use case: Create a list of beverages}

% \begin{center}
%   \begin{tabular}{| l | p{10.75cm} | }
%     \hline
%     Actor        & Restaurant \\
%     \hline
%     Description  &  \\
%     \hline
%     Scenario     &
%     \begin{minipage}[t]{\linewidth}
%       \begin{enumerate}[leftmargin=*,nosep,before=\vspace{-0.575\baselineskip},after=\strut]
%         \item The restaurant employee logs in to the application.
%         \item The restaurant employee clicks a button labeled "Create new menu". 
%         \item The application asks the restaurant employee what kind of menu they would like to create.
%         \item The restaurant employee chooses that they wish to create a list of beverages.
%         \item The restaurant employee specifies beverages of the menu.
%         \item The restaurant employee clicks a "Save menu" button.
%         \item The application saves the menu.
%       \end{enumerate}
%     \end{minipage}
%     \\
%     \hline
%     % Alternatives &
%     % \begin{minipage}[t]{\linewidth}
%     %   \begin{description}[nosep,after=\strut]
%     %     \item [A1:] ...
%     %   \end{description}
%     % \end{minipage}
%     % \\
%     % \hline
%   \end{tabular}
%   \newline
% \end{center}

% \noindent \textbf{22. Use case: Create a stable menu in a foreign language}

% \begin{center}
%   \begin{tabular}{| l | p{10.75cm} | }
%     \hline
%     Actor        & Restaurant \\
%     \hline
%     Description  & A restaurant's management wishes to have \\
%     \hline
%     Scenario     &
%     \begin{minipage}[t]{\linewidth}
%       \begin{enumerate}[leftmargin=*,nosep,before=\vspace{-0.575\baselineskip},after=\strut]
%         \item The restaurant employee logs in to the application.
%         \item The restaurant employee clicks a button labeled "Create new menu". 
%         \item The application asks the restaurant employee what kind of menu they would like to create.
%         \item The restaurant employee selects that they wish to create a stable menu.

%         \item The restaurant employee selects what currency to use in the menu.
%         \item The restaurant employee selects what weight measurement units to use in the menu.

%         \item The restaurant employee clicks a button labeled "Add category".
%         \item The application displays a screen for adding a new category.
%         \item The restaurant employee specifies the name of a new category of the menu in the foreign language.
%         \item The restaurant employee clicks an "Add" button.
%         \item The application adds the category to the menu.
%         \item The restaurant employee repeats steps x to y until they have created all categories of the new menu.
        
%         \item The restaurant employee clicks an "Add item" button.
%         \item The application asks the restaurant employee whether they would like to add a meal or a drink to the menu.
%         \item The restaurant employee selects that they wish to add a meal to the menu. A1 drink
%         \item The application displays a screen for adding a new meal to the menu.
%         \item The restaurant employee specifies the meal's name in the foreign language.
%         \item The restaurant employee specifies the meal's price and weight.
%         \item The restaurant employee clicks a button labeled "Add ingredient".
%         \item The application provides a list of possible ingredients.
%         \item The restaurant employee selects an ingredient from the list and clicks "Add".
%         \item The restaurant employee repeats steps x to y until all ingredients of the meal are added.
%         \item The restaurant employee clicks a button labeled "Add allergen". A2 - the application fills allergens contained in the ingredient automatically based on the provided ingredient
%         \item The application provides a list of possible allergens.
%         \item The restaurant employee selects an allergen from the list and clicks "Add".
%         \item The restaurant employee repeats steps x to y until all allergens of the meal are specified.
%         \item The restaurant employee repeats steps x to y until they have specified all items of the menu.

%         \item The restaurant employee clicks a "Save menu" button.
%         \item The application saves the menu.
%       \end{enumerate}
%     \end{minipage}
%     \\
%     \hline
%     Alternatives &
%     \begin{minipage}[t]{\linewidth}
%       \begin{description}[nosep,after=\strut]
%         \item [A1:] ...
%       \end{description}
%     \end{minipage}
%     \\
%     \hline
%   \end{tabular}
%   \newline
% \end{center}

% \noindent \textbf{x. Use case: Create a new menu}

% \begin{center}
%   \begin{tabular}{| l | p{10.75cm} | }
%     \hline
%     Actor        & Restaurant \\
%     \hline
%     Scenario     &
%     \begin{minipage}[t]{\linewidth}
%       \begin{enumerate}[leftmargin=*,nosep,before=\vspace{-0.575\baselineskip},after=\strut]
%         \item The restaurant employee clicks a button labeled "Create new menu". 
%         \item The application displays a screen for creating a new menu.
%         \item The restaurant employee creates categories of the new menu. \textbf{A1}
%         \item The restaurant employee specifies items, along with allergens contained in them, and assign them to the categories of the new menu.
%         \item The restaurant employee clicks a "Save" button and the application saves the menu.
%       \end{enumerate}
%     \end{minipage}
%     \\
%     \hline
%     Alternatives &
%     \begin{minipage}[t]{\linewidth}
%       \begin{description}[nosep,after=\strut]
%         \item [A1:] The restaurant employee skips the creation of categories and only specifies the items of the menu. The restaurant employee then continues with step 5.
%       \end{description}
%     \end{minipage}
%     \\
%     \hline
%   \end{tabular}
%   \newline
% \end{center}

% \noindent \textbf{x. Use case: Select what currency to use in a menu}

% \begin{center}
%   \begin{tabular}{| l | p{10.75cm} | }
%     \hline
%     Actor        & Restaurant \\
%     \hline
%     Description  & A restaurant employee has created an empty menu and wants to set what currency to use in a menu. \\
%     \hline
%     Scenario     &
%     \begin{minipage}[t]{\linewidth}
%       \begin{enumerate}[leftmargin=*,nosep,before=\vspace{-0.575\baselineskip},after=\strut]
%         \item The restaurant employee clicks a button labeled "Choose currency for the menu".
%         \item The application provides a predefined list of currencies. 
%         \item The restaurant employee selects the desired currency and clicks an "Ok" button.
%         \item The application sets the selected currency for the menu.
%       \end{enumerate}
%     \end{minipage}
%     \\
%     \hline
%   \end{tabular}
%   \newline
% \end{center}

% \noindent \textbf{x. Use case: Select what weight units to use in a menu}

% \begin{center}
%   \begin{tabular}{| l | p{10.75cm} | }
%     \hline
%     Actor        & Restaurant \\
%     \hline
%     Description  & A restaurant employee has created an empty menu and wants to set what weight units to use in it. \\
%     \hline
%     Scenario     &
%     \begin{minipage}[t]{\linewidth}
%       \begin{enumerate}[leftmargin=*,nosep,before=\vspace{-0.575\baselineskip},after=\strut]
%         \item The restaurant employee clicks a button labeled "Choose weight units".
%         \item The application provides a predefined set of weight units.
%         \item The restaurant employee selects the desired weight unit and clicks an "Ok" button.
%         \item The application sets the selected weight unit for the menu.
%       \end{enumerate}
%     \end{minipage}
%     \\
%     \hline
%   \end{tabular}
%   \newline
% \end{center}

% \noindent \textbf{x. Use case: Add a category to a menu}

% \begin{center}
%   \begin{tabular}{| l | p{10.75cm} | }
%     \hline
%     Actor        & Restaurant \\
%     \hline
%     Description  & A restaurant employee has created an empty menu and wants to add a category to it. \\
%     \hline
%     Scenario     &
%     \begin{minipage}[t]{\linewidth}
%       \begin{enumerate}[leftmargin=*,nosep,before=\vspace{-0.575\baselineskip},after=\strut]
%         \item The restaurant employee clicks a button labeled "Add category".
%         \item The restaurant employee specifies the name of the category.
%         \item The application adds the category to the menu.
%       \end{enumerate}
%     \end{minipage}
%     \\
%     \hline
%   \end{tabular}
%   \newline
% \end{center}

% \noindent \textbf{x. Use case: Specify allergens contained in an item of a menu}

% \begin{center}
%   \begin{tabular}{| l | p{10.75cm} | }
%     \hline
%     Actor        & Restaurant \\
%     \hline
%     Description  & A restaurant's employee is creating a menu. They have added an item to the menu and want to specify allergens contained in it. \\
%     \hline
%     Scenario     &
%     \begin{minipage}[t]{\linewidth}
%       \begin{enumerate}[leftmargin=*,nosep,before=\vspace{-0.575\baselineskip},after=\strut]
%         \item The restaurant employee clicks a button labeled "Add allergen". \textbf{A1}
%         \item The application provides a predefined list of possible allergens.
%         \item The restaurant employee selects an allergen from the list and clicks "Add".
%         \item The restaurant employee repeats steps x to y until all allergens of the item are added.
%       \end{enumerate}
%     \end{minipage}
%     \\
%     \hline
%     Alternatives &
%     \begin{minipage}[t]{\linewidth}
%       \begin{description}[nosep,after=\strut]
%         \item [A1:] The application adds allergens contained in the ingredient automatically based on the provided ingredient (see x. Use case: Add an item to a menu).
%       \end{description}
%     \end{minipage}
%     \\
%     \hline
%   \end{tabular}
%   \newline
% \end{center}

% \noindent \textbf{x. Use case: Add an item to a menu}

% \begin{center}
%   \begin{tabular}{| l | p{10.75cm} | }
%     \hline
%     Actor        & Restaurant \\
%     \hline
%     Description  & A restaurant's employee has created an empty menu and wants to add an item to it. \\
%     \hline
%     Scenario     &
%     \begin{minipage}[t]{\linewidth}
%       \begin{enumerate}[leftmargin=*,nosep,before=\vspace{-0.575\baselineskip},after=\strut]
%         \item The restaurant employee clicks an "Add item" button.
%         \item The application displays a screen for adding an item to a menu.
%         \item The restaurant employee specifies the item's name, price and amount.
%         \item The restaurant employee clicks a button labeled "Add ingredient".
%         \item The application provides a predefined list of ingredients.
%         \item The restaurant employee selects an ingredient from the list and clicks "Add". \textbf{A1}
%         \item The restaurant employee repeats steps x to y until all ingredients of the item are added.
%         \item The restaurant employee adds allergens contained in the item (see x. Use case: Specify allergens contained in an item of a menu).
%       \end{enumerate}
%     \end{minipage}
%     \\
%     \hline
%     Alternatives &
%     \begin{minipage}[t]{\linewidth}
%       \begin{description}[nosep,after=\strut]
%         \item [A1:] The restaurant employee also specifies the amount of the ingredient contained in the item. The restaurant employee continues with step x.
%       \end{description}
%     \end{minipage}
%     \\
%     \hline
%   \end{tabular}
%   \newline
% \end{center}

% \noindent \textbf{x. Use case: Add my restaurant's information to a menu}

% \begin{center}
%   \begin{tabular}{| l | p{10.75cm} | }
%     \hline
%     Actor        & Restaurant \\
%     \hline
%     Description  & A restaurant's employee has created an empty menu and wants to add basic information about the restaurant. \\
%     \hline
%     Scenario     &
%     \begin{minipage}[t]{\linewidth}
%       \begin{enumerate}[leftmargin=*,nosep,before=\vspace{-0.575\baselineskip},after=\strut]
%         \item The restaurant employee clicks a button labeled "Add restaurant information".
%         \item The application displays a screen for adding basic restaurant information.
%         \item The restaurant employee specifies the restaurant's name, address and phone number.
%         \item The restaurant employee clicks an "Add" button.
%         \item The application adds the specified information to the menu. 
%       \end{enumerate}
%     \end{minipage}
%     \\
%     \hline
%   \end{tabular}
%   \newline
% \end{center}

% \noindent \textbf{x. Use case: Specify where should the application store my data }

% \begin{center}
%   \begin{tabular}{| l | p{10.75cm} | }
%     \hline
%     Actor        & Restaurant \\
%     \hline
%     Description  &  \\
%     \hline
%     Scenario     &
%     \begin{minipage}[t]{\linewidth}
%       \begin{enumerate}[leftmargin=*,nosep,before=\vspace{-0.575\baselineskip},after=\strut]
%         \item ...
%         \item ... \textbf{A1}
%         \item ...
%       \end{enumerate}
%     \end{minipage}
%     \\
%     \hline
%     Alternatives &
%     \begin{minipage}[t]{\linewidth}
%       \begin{description}[nosep,after=\strut]
%         \item [A1:] ...
%       \end{description}
%     \end{minipage}
%     \\
%     \hline
%   \end{tabular}
%   \newline
% \end{center}