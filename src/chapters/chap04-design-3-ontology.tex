\section{Ontology}
Choose Well defines its own RDF vocabulary. 
\todo[inline]{add figure number}
Fig. x contains the created ontology as a UML class diagram. 
The diagram uses these prefixes:
\todo[inline]{change font of prefixes to code}
\todo[inline]{add arrow symbols for prefixes}
\todo[inline]{add hashtags: /blob/main/choosewellHASHTAG http://www.w3.org/2000/01/rdf-schemaHASHTAG}
\begin{itemize}[noitemsep,nolistsep]
  \item cw: https://github.com/JiriResler/solid-choose-well-ontology/ \newline blob/main/choosewell for the Choose Well ontology.
  \item schema: \textbf{http://schema.org/} for the Schema.org general vocabulary. 
  \item rdfs: \textbf{http://www.w3.org/2000/01/rdf-schema} for the RDF Schema which provides a data-modelling vocabulary for RDF data.
  \item wikidata: \textbf{http://www.wikidata.org/entity/} for describing entities using the Wikidata knowledge base.
\end{itemize}

\begin{figure}[h]
  \centering
  \includegraphics[width=\linewidth]{master-thesis/img/design-ontology.pdf}
  \caption{The Choose Well ontology}
\end{figure}

More on design \\
The QR code will link to the URL of the application and will provide it with the IRI of the menu just like in the requirement 1.7.