\section{User requirements vocabulary}
It is good practice to use language consistently throughout all requirements, and that is why we are going to resolve what the keywords in requirements mean.
If a requirement states that the application \textbf{shall} do something, then it means that the requirement is mandatory for the application and has to be addressed in the design phase which follows after this chapter. 
The word \textbf{should} is for desired requirements which are not critical for the application's functionality.
Last but not least, usage of the word \textbf{must} indicates a domain constraint.
Also, we will substitute the user role restaurant for \emph{restaurant employee} where it will be more convenient.
In the next two sections, we are going to specify all user requirements. 

\section{Functional requirements}
We will look at requirements from the application's user point of view.
That is why we will organize the requirements based on which user role defines them.
If a requirement is defined by both user roles, it will be listed outside the role specific sections.
The following list of functional and non-functional requirements is inspired by an interview done with a restaurant\footnote{\url{https://pizzabudca.sk/}  \label{fnlabel}} employee.

\subsection{Guest requirements}
A guest is expected to be logged in if not stated otherwise in a requirement.

\begin{description}
    \item [Req. 1.1:] The application shall enable a restaurant guest to specify what allergies they have.
    \item [Req. 1.2:] The application shall enable a restaurant guest to specify what diets they are on.
    \item [Req. 1.3:] The application should enable a restaurant guest to specify what food and beverages they like and dislike.
    \item [Req. 1.4:] The application shall enable a restaurant guest to view a menu of the restaurant which is personalized based on the guest's profile.

    \emph{Rationale:} A personalized menu highlights or hides items of the menu in a way that the guest can quickly choose what they want to eat. 
    \item [Req. 1.5:] The application should be able to sort meals in a menu by whether a viewing guest can eat them according to their profile.

    \emph{Rationale:} This will allow the application to highlight the meals of a menu which the guest can eat according to their profile.
    \item [Req. 1.6:] The application should be able to filter out meals of a menu which a viewing guest cannot eat according to their profile.
    \item [Req. 1.7:] A guest shall be able to load a menu by specifying the IRI of the menu.
    \item [Req. 1.8:] A guest should be able to load a menu by scanning a QR code printed on the menu.

    \emph{Rationale:} The QR code will link to the application and will provide it with the IRI of the printed menu.
    \item [Req. 1.9:] A guest should be able to specify the IRI of a restaurant and browse its menus.
    \item [Req. 1.10:] The application should enable a guest to mark a restaurant as favorite.
    
    \emph{Rationale:} A guest should have a set of their favorite restaurants.
    \item [Req. 1.11:] A guest should be able to view currently served meals by their favorite restaurants.
    \item [Req. 1.12:] A guest should be able to specify which Solid pod to use for storing their data.

    \emph{Rationale:} A guest can have multiple Solid pods associated with their WebID and might want to choose which one should the application use.
    \item [Req. 2.11:] A non-authenticated guest should be able to do reqs. x, y, z. 

    \emph{Rationale:} None of these requirements require the guest to be logged in. The application should be also useful for people without a profile.
\end{description}

\subsection{Restaurant requirements}
\begin{description}
    \item [Req. 2.1:] A restaurant employee shall be able to create a menu for a specific day.

    \emph{Rationale:} Restaurant often have daily menus.
    \item [Req. 2.2:] A restaurant employee shall be able to create a stable menu.

    \emph{Rationale:} Most restaurants have a stable menu which is valid every day and does not change often.
    \item [Req. 2.3:] A restaurant employee should be able to create a list of beverages.

    \emph{Rationale:} Some restaurants have a separate menu for meals and for drinks.
    \item [Req. 2.4:] A restaurant employee should be able to re-use a previously created menu when creating menu for a day.
    \item [Req. 2.5:] A restaurant employee should be able to print a menu.

    \emph{Rationale:} A printed menu can be put on tables.
    \item [Req. 2.6:] A printed menu should contain a QR code which will link a guest to the application.

    \emph{Rationale:} The QR code will provide the application with the IRI of the menu.
    \item [Req. 2.7:] The application shall enable a restaurant employee to edit a previously created menu.
    \item [Req. 2.8:] The application shall enable a restaurant employee to delete a previously created menu.
    \item [Req. 2.9:] A restaurant employee should be able to set a daily menu to repeat periodically for a certain day of the week.

    \emph{Rationale:} A daily menu can be the same for a certain day of the week.
    \item [Req. 2.10:] A restaurant employee should be able to specify categories and subcategories of a menu.

    \emph{Rationale:} A menu typically consists of categories like soups, appetizers, desserts etc. These categories can have subcategories, for instance a category "Drinks" can have subcategories "Wine" and "Non-alcoholic".
    \item [Req. 2.11:] Each item of a menu should belong to either none or exactly one category.
    \item [Req. 2.12:] A meal in a menu should contain an ID, label, price, weight, ingredients and allergens.
    \item [Req. 2.13:] A restaurant employee should be able to specify weights of ingredients contained in a meal.
    \item [Req. 2.14:] An item on a list of beverages should contain the name of a drink, amount and price.
    \item [Req. 2.15:] A menu should optionally contain a restaurant's name, address and phone number.
    \item [Req. 2.16:] A restaurant employee should be able to use a previously created menu as a template for creating a new menu.
    \item [Req. x:] A restaurant employee should be able to specify what currency to use for prices of items of a menu.
    \item [Req. x:] A restaurant employee should be able to specify what weight and volume units to use for items of a menu.
    
    \emph{Rationale:} Req. x states that a food should have weight specified.
    \item [Req. x:] A restaurant employee should be able to set that a menu shall use allergen labels for displaying allergen information.

    \emph{Rationale:} Allergens can be listed within an item's description.
    \item [Req. x:] A restaurant employee should be able to set that a menu shall use numbers for displaying allergen information.

    \emph{Rationale:} A menu item can contain numbers which represent allergens. This is more concise than using labels from req. x. 
    \item [Req. 2.17:] A restaurant employee should be able to specify which Solid pod to use for storing the restaurant's data.

    \emph{Rationale:} A restaurant can have multiple Solid pods associated with its WebID, and the restaurant's management might want to choose which pod should the application use.
\end{description}

\section{Non-functional requirements}
\begin{description}
    \item [Req. 3.1:] Each item of a menu must have specified allergens contained in it.
    
    \emph{Rationale:} At least in the EU, there are laws\footnote{\url{https://eur-lex.europa.eu/legal-content/EN/ALL/?uri=celex\%3A32011R1169}  \label{fnlabel}} which mandate restaurants to list allergens contained in food they serve.
    \item [Req. x:] A menu which uses allergen numbers from requirement x must contain a legend explaining what allergen each number represents.
    \item [Req. 3.2:] The application should have responsive user interface and work on mobile devices as well as on desktops.
    \item [Req. 3.2:] The application should be compatible with the latest versions of all of the commonly used browsers.
    \item [Req. 3.3:] The application shall have a user tutorial explaining its functionality.
    \item [Req. 3.4:] The application should contain a help screen for new users with information about how to log in.

    \emph{Rationale:} The new user may not be familiarized with the Solid technology.
    \item [Req. 3.5:] The application should have an English, Czech and Slovak user interface.

    \emph{Rationale:} We would like for the application to be used internationally within the European Union.
\end{description}

\vspace*{\fill}