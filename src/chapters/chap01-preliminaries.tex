\setlength{\epigraphwidth}{.5\textwidth}
\setlength\epigraphrule{0pt}

\chapter{Preliminaries}
This chapter explains to the reader terms which are used in this thesis.
We introduce concepts on which we build our application.

\todo[inline]{add references}
\section{Semantic Web}
\epigraph{\textit{``The Web as I envisaged it, we have not seen it yet. The future is still so much bigger than the past."}}{--- Sir Tim Berners-Lee}

According to the World Wide Web Consortium(link) the Semantic Web is an extension of the Web as we know it today.
Currently, the Web consists of interconnected web pages and is called a Web of documents.
This view is natural to humans, but computers do not understand what is on a web page based solely on its content. 
A computer sees a text or an image, but does not know what it means in the language of humans.
It would help if a web page, or any other resource on the internet, had a single source of truth about what it is and what it is connected to.
This is achievable by implementing a Web of data where all data is interconnected.
Data on the Web is assigned its meaning or semantics, thus the term Semantic Web.

\subsection*{Linked data}
If data is to be connected, we need tools and specifications for how to implement it.
We need a uniform data format and also a query language.
For a data format, RDF is used.
Data in this format is expressed in a form of triples.
It can be viewed as a mathematical graph where data are nodes and links between them are edges of the graph.
When data is stored as a graph, some queries are much easier to write and are more effective than in traditional SQL databases.
A query language which is used is called SPARQL.
Ontologies are a way of describing metadata about data.


\section{The Solid project}
Solid makes it possible for end-users to have control over their data.
It provides Pods, which act as personal data storage.
A user can set what applications and who can see and manipulate with their data.
