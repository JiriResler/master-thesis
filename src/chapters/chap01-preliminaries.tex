\setlength{\epigraphwidth}{.5\textwidth}
\setlength\epigraphrule{0pt}

\chapter{Preliminaries}
This chapter aims to explain terms which may be new to the reader which are used later in this thesis.
We will discuss problems of today's state of the Web and what can be done to fix it.

\section{Semantic Web}
\epigraph{\textit{``The Web as I envisaged it, we have not seen it yet. The future is still so much bigger than the past."}}{--- Sir Tim Berners-Lee}
% Linked Data
% RDF
% RDF vocabulary



\section{The Solid project}
% Solid pod
% WebID
% Solid provider

\section*{Origin}
Solid is a mid-course correction for the Web by its inventor, Sir Tim Berners-Lee. It realizes Tim's original vision for the Web as a medium for the secure, decentralized exchange of public and private data.

\section*{Why fix the Web?}
The first web browser was also an editor. The idea being that not only could everyone read content on the web, but they could also help create it. It was to be a collaborative space for everyone.

However, when the first browser that popularized the web came along, called Mosaic, it included multimedia and editing was taken out. It was considered too difficult a problem. This change was the first curtailing of the web's promise and spawned an effort led by Tim and others to get the write functionality back. It was dubbed the "read-write web" and led to Richard McManus' seminal article published in 2003.

The issue with writing data, as Wikipedia and others have learned, is that you need a degree of control over who can write what. That means you need to have permissions to dictate what individuals can do to the data. And to have permissions you need to have a system for identity - a way of uniquely authenticating that an individual is who they purport to be.

Of course the social networks solved this problem within their own systems using their own specific identity and access control, but these were not standard, not interoperable, and gave you no choice in what applications you could use to access that data. You had to have your entire personal or professional life within one silo for it to work. And since the Web is ubiquitous, these silos exist across the data spectrum, from social and medical to financial and civil.

When your data is siloed away from you:

- You have hardly any visibility into what is being retained.
- You have little control over how it is used, or who is using it.
- You have little choice in which applications you can use to access it.
- It is hard to use as a cohesive unit, specifically because it is siloed, scattered across proprietary vendors, interfaces, and data formats.
All of these factors combine to make it very hard to access all of your own data, and put it to work on your behalf.