\chapter{Design}
Now we will design how to implement what we have defined in the analysis chapter.
Our application is a single-page web application.

\section{Architecture}
\begin{figure}[h]
  \centering
  \includegraphics[width=0.62\linewidth]{master-thesis/img/architecture_data_flow.pdf}
  \caption{Application's data flow diagram}
\end{figure}

\section{Technological stack}
There are various options out there for what technologies we can use to implement our application.
This section contains an overview of available tools and which ones we chose and why.

\subsection{User interface}
We need a UI so that a user can interact with the application.
We need to be able to create an interactive user interface.
We need to have state for logged in user.
We need to fetch data from the internet.
We need to integrate with Solid.

\subsection*{Considered technologies}
Nowadays, there are many frameworks for building web applications.
The most popular options are Angular, React and Vue.

\subsubsection*{Angular}
Angular is a platform and framework for building single-page client applications using HTML and TypeScript. 
It implements its core and optional functionality as a set of TypeScript libraries. 
The basic building blocks of the Angular framework are Angular components that are organized into \emph{NgModules}. 
NgModules collect related code into functional sets; an Angular application is defined by a set of NgModules.

\subsubsection*{React.js}
React is a JavaScript library for rendering user interfaces.
Applications written in React are built from modular and reusable pieces called components.
React components receive data and return what should appear on the screen. 
They can be passed new data in response to an interaction, like when the user types into an input. 
React will then update the screen to match the new data.
A notable feature is the use of a virtual Document Object Model, or Virtual DOM. 
React creates an in-memory data-structure cache, computes the resulting differences on a re-render, and then updates the browser's displayed DOM efficiently. 
This selective rendering provides a major performance boost.

\subsubsection*{Vue.js}
Vue is a JavaScript framework for building user interfaces. 
It builds on top of standard HTML, CSS, and JavaScript and provides a declarative and component-based programming model.
Vue uses an HTML-based template syntax that allows programmers to declaratively bind the rendered DOM to the underlying component instance's data.
Under the hood, Vue compiles the templates into highly-optimized JavaScript code. 
Combined with the reactivity system, Vue can intelligently figure out the minimal number of components to re-render and apply the minimal amount of DOM manipulations when the app state changes.

\subsection*{Chosen technology}
We decide to choose React for developing the UI of our application.
One of the reasons for this decision is that Solid is written for React.
There exists a Solid React SDK which can help us to work with some of the core Solid principles.
Ract also allows us to work with components as functions.

\subsection{nasadenie}
%  Vite
% alternatives: create react app, webpack, Next.js

\subsection{client-side routing}
% react router if used

\subsection{Programming language}
We need to implement the application.

\subsection*{Considered technologies}
As for programming languages, we considered two options. 
JS, TS

\subsubsection*{JavaScript}
JavaScript is a lightweight, interpreted, or just-in-time compiled programming language with first-class functions. 
It is most well-known as the scripting language for Web pages. 
JavaScript is a prototype-based, multi-paradigm, single-threaded, dynamic language, supporting object-oriented, imperative, and declarative---e.g. functional programming styles.

\subsubsection*{TypeScript}
TypeScript is a superset of JavaScript. 
It provides features such as optional static typing, classes, interfaces, and generics. 
The goal of TypeScript is to help catch mistakes early through its type system and make JavaScript development more efficient. 
One of the big benefits is enabling IDEs to provide a richer environment for spotting common errors as the programmer types their code.

\subsection*{Chosen technology}


\subsection{Package manager}
We need a tool to manage our application's dependencies.
A package manager can seamlessly handle installing and uninstalling of packages which is another name for JavaScript libraries.

\subsection*{Considered technologies}
For package management in our application we consider two options: npm and yarn.

\subsubsection*{npm}
npm is a package manager for the JavaScript programming language.
npm is used to fetch any packages that an application needs for development, testing, and/or production, and may also be used to run tests and tools used in the development process.
It consists of a command line client, also called npm, and an online database of packages, called the npm registry. 
The registry is accessed via the client, and the available packages can be browsed and searched via the npm website.

\subsubsection*{yarn}
A successful and popular alternative package manager is Yarn. 
Yarn resolves the dependencies using a different algorithm that can mean a faster user experience.
More specifically, yarn can download packages in parallel to maximize network utilization.

\subsection*{Chosen technology}
Although Yarn is just as good as npm for handling packages, the latter is more widely used.
For this reason we decide to use npm for our application. 

\subsection{Responsive design}
We need our application to be responsive to various device screens.
We need a mobile-first approach

\subsection*{Considered technologies}
Bootstrap, MUI
\subsubsection*{Bootstrap}

\subsubsection*{Material UI}

\subsection*{Chosen technology}



\subsection{Persistence}
We need to store and later read data.
% Solid pods
\subsection*{Considered technologies}

\subsubsection*{}

\subsubsection*{}

\subsection*{Chosen technology}

\subsection{Testing}
We need to test our application in order to prevent bugs during implementation.
% https://legacy.reactjs.org/docs/testing-environments.html
\subsection*{Considered technologies}

\subsubsection*{}

\subsubsection*{}

\subsection*{Chosen technology}

\subsection{Documentation}
We need to capture how our application works for users and future developers.
% (GitHub markdown)
\subsection*{Considered technologies}

\subsubsection*{}

\subsubsection*{}

\subsection*{Chosen technology}
