\chapter{Problem analysis}
We first need to analyze our problem domain.
This will be achieved by identifying different kinds of user roles, specifying user requirements, creating use cases, and describing those use cases via use case scenarios.

\section{User roles}
There are two types of users who will come in contact with the application.
The first type is a \textbf{restaurant}, more specifically its employee who is responsible for creating menus.
This user needs to have a basic understanding of how to use a computer or smartphone and a web browser.
They also need to know how to register for a WebID and a Solid pod.
The application will help restaurant employees to create menus online, and will let them specify allergens contained in each meal of a menu.

The second type of users are the restaurant \textbf{guests}.
They, too, need to have a basic understanding of how to use a computer or smartphone and a web browser, and need to know how to register for a WebID and a Solid pod.
The application will enable a restaurant guest to create a personal profile where they will specify their food preferences, including allergies and diets.
When a guest will visit a restaurant, the application will enable them to filter out meals in the restaurant's menu which they are automatically not interested in ordering based on their profile.

\section{Functional requirements}

\section{Non-functional requirements}

\section{Use cases}

\section{Use case scenarios}