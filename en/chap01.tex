\chapter{Solid}

An~example citation: \cite{Andel07}

\section{Title of the first subchapter of the first chapter}

\section{Title of the second subchapter of the first chapter}

Most of the web applications today share a business model where users get functionality without having to pay for it in exchange for their data.
This data---which may contain personal information---is then passed to a third-party company that uses it to show personalized advertisement to users. 
The result of this is that a few corporate companies have obtained huge amounts of user data over the years which may end up for example in a data breach. 
This centralized approach also slows innovation as it is hard for new players to compete with these companies because data and applications are inseparable. 
The solution to these issues is to re-decentralize the web as it was originally meant to be. 
One of the technologies to achieve this goal is Solid. 
It specifies how to provide users with Solid pods, which are personal data stores. 
The owner of a pod defines who or which applications have access to it. 
This makes users take back control over their data and applications become simple views for that data. 
This also means that data and applications become two separate markets so anyone with a good application can be successful. 
This thesis aims to support the idea of re-decentralization of the web by documenting the process of the creation of a web application which uses Solid pods for storing user data. 
The application helps people who are on a diet or have allergies. 
Its user is able to browse personalized restaurant menus which are automatically adjusted based on their profile.